%! TEX program = lualatex
% find: (\$((?!\$).)+\$)
%replace: \E{\1}

%%%%%%%%%%
% BEAMER %
%%%%%%%%%%

\RequirePackage{luatex85}

\PassOptionsToPackage{luatex}{hyperref}

\documentclass[luatex]{beamer}
%\documentclass[handout]{beamer}

\usetheme{metropolis}

	\newcommand{\bigq}{\mathbb{Q}}
\newcommand{\bb}{\bfseries\large}
\newcommand{\class}{\mathcal}
\newcommand{\mbi}[1]{\textbf{\em{#1}}}
\newcommand{\bi}[1]{\textbf{\textit{#1}}}
\newcommand{\ita}[1]{\textit{#1}}
\newcommand{\p}{\cdot}
\newcommand{\req}{\emptyset}
\newcommand{\init}{\mathcal{J}}
\newcommand{\power}[1]{\class{P}(#1)}
\newcommand{\bigpower}[2]{\class{P}^{#1}(#2)}
\newcommand{\symetric}{\bigtriangleup}
\newcommand{\gx}[3]{\Gamma_{(#1,#2)}(#3)}
\newcommand{\bd}[3]{\Delta_{(#1,#2)}(#3)}
\newcommand{\bp}[3]{\Phi_{(#1,#2)}(#3)}
\newcommand{\eab}{\eta_{\alpha \beta}}
\newcommand{\sab}{\sigma_{\alpha\beta}}
\newcommand{\dab}{\class{D}_{\alpha\beta}}
\newcommand{\cab}{\class{C}_{\alpha\beta}}
\newcommand{\C}[1]{\class{#1}}
\newcommand{\ol}{\overline}
\newcommand{\into}{\longrightarrow}
\newcommand{\ifff}{\Longleftrightarrow}
\newcommand{\imply}{\Rightarrow}
\newcommand{\since}{\Longleftarrow}
\newcommand{\la}{\langle}
\newcommand{\ra}{\rangle}
\newcommand{\andd}{\wedge}
\newcommand{\orr}{\vee}
\newcommand{\ft}{(\C F \C T)_Q}
\newcommand{\pathopen}[3]{#1_{(\ol{#2},#3)}}
\newcommand{\fpi}[3]{#1_*(\Pi(#2,#3))}
\newcommand{\xpath}[2]{(\ol {#1},#2)}
\newcommand{\qn}[2]{\Lambda(#1,#2)}
\newcommand{\jh}{\class J \class H}
\newcommand{\bos}{\boldsymbol}
\newcommand{\f}{\mathbb}
\newcommand{\rss}{\la\psi_i:i\in \lambda\ra}
\newcommand{\ito}[1]{\overset{#1}{\longrightarrow}}
\newcommand{\pcomplex}[2]{\class #1(\class #2)}
\newcommand{\then}{\Rightarrow}
\newcommand{\ud}{{ {up-down-automaton}} }

\newcommand{\map}[1]{\underset{#1}{\mapsto}}
\newcommand{\mapi}[1]{\underset{#1}{\overset{i}{\mapsto}}}
\newcommand{\mapstar}[1]{\underset{#1}{\overset{*}{\mapsto}}}
\newcommand{\mapk}[2]{\underset{#1}{\overset{#2}{\mapsto}}}

\newcommand{\eql}{\equiv_L}
\newcommand{\der}[1]{\underset{#1}{\imply}}
\newcommand{\deri}[1]{\underset{#1}{\overset{i}{\imply}}}
\newcommand{\derstar}[1]{\underset{#1}{\overset{*}{\imply}}}
\newcommand{\derk}[2]{\underset{#1}{\overset{#2}{\imply}}}
\newcommand{\blank}{\bar{b}}
\newcommand{\R}[1]{#1}
\newcommand{\say}[1]{\color{blue}{#1}\color{black}}



% Fix the title separator
\makeatletter
\setbeamertemplate{title separator}{
	\bgroup
	\bodydir TLT
	\begin{tikzpicture}
	\fill[fg] (0,0) rectangle (\textwidth, \metropolis@titleseparator@linewidth);
	\end{tikzpicture}%
	\egroup
	\par%
}
\makeatother

% Fix the progress bar and make it progress from right to left
\makeatletter
\setbeamertemplate{progress bar in section page}{
	\setlength{\metropolis@progressonsectionpage}{%
		\textwidth * \ratio{\insertframenumber pt}{\inserttotalframenumber pt}%
	}%
	\bgroup
	\bodydir TLT
	\begin{tikzpicture}
	\fill[bg] (0,0) rectangle (\textwidth, \metropolis@progressonsectionpage@linewidth);
	\fill[fg] (\textwidth,0) rectangle (\textwidth-\metropolis@progressonsectionpage, \metropolis@progressonsectionpage@linewidth);
	\end{tikzpicture}%
	\egroup
}
\makeatother

% A more logical title page, juxtaposing `institute` to `author`. This modification is not necessary for Hebrew support.
\makeatletter
\setbeamertemplate{title page}{
	\begin{minipage}[b][\paperheight]{\textwidth}
		\ifx\inserttitlegraphic\@empty\else\usebeamertemplate*{title graphic}\fi
		\vfill%
		\ifx\inserttitle\@empty\else\usebeamertemplate*{title}\fi
		\ifx\insertsubtitle\@empty\else\usebeamertemplate*{subtitle}\fi
		\usebeamertemplate*{title separator}
		\ifx\beamer@shortauthor\@empty\else\usebeamertemplate*{author}\fi
		\ifx\insertinstitute\@empty\else\usebeamertemplate*{institute}\fi
		\ifx\insertdate\@empty\else\usebeamertemplate*{date}\fi
		\vfill
		\vspace*{1mm}
	\end{minipage}
}
\setbeamertemplate{author}{\vspace*{2em}\insertauthor\par\vspace*{0.25em}}
\setbeamertemplate{institute}{\insertinstitute\par}
\setbeamertemplate{date}{\vspace*{3mm}\insertdate\par}
\makeatother




%%%%%%%%%
% Babel %
%%%%%%%%%

\usepackage[nil,bidi=basic-r]{babel}
\babelprovide[import=he,main]{hebrew}
\babelprovide[import=en-GB]{english}

% For some reason Babel’s `\babelfont` doesn’t work
\setsansfont[Script=Hebrew]{Open Sans Hebrew}
\setmonofont{Fira Mono}
\renewcommand{\H}[1]{\foreignlanguage{hebrew}{\fontspec[Script=Hebrew]{Open Sans Hebrew}#1}}
\newcommand{\E}[1]{\foreignlanguage{english}{\fontspec{Open Sans}#1}}
\newcommand{\LR}[1]{{‏\textdir TLT #1}}




%%%%%%%%
% MISC %
%%%%%%%%

\usepackage{metalogo, fancyvrb}
\usepackage{cancel}
\newcommand{\smallurl}[1]{{\footnotesize\url{#1}}}



%%%%%%%%%%%%
% DOCUMENT %
%%%%%%%%%%%%

\begin{document}
	\title{הפרד ומשול}
	\subtitle{מפגש 7}
	%\author{}
	%\institute{}
	\date{}
	\begin{frame}{לפני הכל בואו נדבר על:}
	\pause
	\begin{itemize}[<+->]
		\item משפט 4.17
		\item הוכחה ע"י הרצה של אלגוריתם
	\end{itemize}
\end{frame}
	\begin{frame}{}
\begin{center}
	\includegraphics[width=10cm]{imgs/trees}
	\end{center}
\end{frame}

\begin{frame}{}
\begin{center}
	\includegraphics[width=10cm]{imgs/trees2}
\end{center}
\end{frame}
	
	\frame{\titlepage}
	\begin{frame}{שבוע שעבר סיימנו}
	\pause
	
	\begin{itemize}[<+->]
		\item פרק של אלגוריתמים חמדניים
		\begin{itemize}[<+->]
\item מסלול קל ביותר - האלגוריתם של דייקסטרא
\item מציאת עץ פורש מינימלי 
\begin{itemize}[<+->]
			\item אלגוריתם של פרים
	\item האלגוריתם של קרוסקל
\end{itemize}
			\item  דחיסת נתונים - קידוד הופמן
		\end{itemize}
	\end{itemize}
\end{frame}

\begin{frame}{התוכנית להיום}
\pause
\begin{itemize}[<+->]
	\item הצגת הרעיון של הפרד ומשול
	\item בעיית פשוטה - מיון מיזוג
	\item ספירת היפוכים בפרמוטציה
	\item הקדמה בנושא FFT
\end{itemize}
\end{frame}

\begin{frame}{הפרד ומשול - הרעיון}
\pause
\begin{itemize}[<+->]
	\item  חלק את הבעיה ל-2 בעיות
	\item פתור כל תת בעיה רקורסיבית
	\item אחד את תתי הבעיות
\end{itemize}
\pause
בפרק זה אנחנו נעסוק בהרבה בעיות אשר אנו מכירים כבר אלגוריתמים יעילים עבורן (שרצים בזמן פולינומי). טכניקת הפרד ומשול תאפשר לנו לבנות אלגוריתם עם זמן ריצה טוב יותר.
\end{frame}
\begin{frame}{דוגמא - מיון מיזוג}
\pause
\begin{itemize}[<+->]
	\item הבעיה: יש למיין מערך
	\item פתרון:
	\pause
	\begin{itemize}[<+->]
		\item נחלק מערך לשתי מערכים שווי גודל
		\item נמיין רקורסיבית כל חצי
		\item נאחד
	\end{itemize}
	\item חלוקה ואיחוד בזמן לינארי בגודל הקלט
	\item נקבל זמן ריצה של:
	$$T(n)=2T(n/2) +cn$$
	זמן הריצה יהיה \E{$nlog(n)$}
\end{itemize}
\end{frame}
\begin{frame}{איך נאחד שני מערכים ממיונים? (בזמן לינארי)}

\end{frame}
\begin{frame}{ניתוח זמן של נוסחאות נסיגה}
\pause
\begin{itemize}[<+->]
	\item ניתן לחשב נוסחאות נסיגה ע"י אינדוקציה או הצבה (הסבר במבוא של פרק 5 בספר למי שרוצה להזכר)
	\item כמה נוסחאות נסיגה שצריך להכיר:
	\pause
	\begin{itemize}[<+->]
		\item זמן ריצה \E{$nlog(n)$}
		$$T(n)=2T(n/2) +cn$$
		\item זמן ריצה לוגרתמי
		$$T(n)=T(n/2) +1$$
		\item זמן ריצה מעריכי
		$$T(n)=2T(n-1) +cn$$
	\end{itemize}
\end{itemize}
\end{frame}
\begin{frame}{ספירת היפוכים בפרמוטציה}
\pause
\begin{itemize}[<+->]
	\item בהינתן סידרה \E{$a_1,a_2,\ldots,a_n$} חשב כמה זוגות יש כך ש \E{$i>j$} וגם \E{$a_i<a_j$}
	\item בייצוג גרפי כמה חיתוכים על \E{$n$} מספרים
	\begin{center}
		\includegraphics[width=6cm]{imgs/hipuhim}
		\end{center}
\end{itemize}
\end{frame}
\begin{frame}{האלגוריתם}
\pause
\begin{itemize}[<+->]
	\item אם הרשימה היא בגודל 1 - החזר אין היפוכים
	\item חלק את הפרמוטציה לשני חלקים \E{$A,B$}
	\item ספור היפוכים ומיין כל תת רשימה בצורה רקורסיבית
	\item אחד את התוצאות של \E{$A$} ו-\E{$B$} יחד עם מספר ההיפוכים בינהם.
\end{itemize}
\end{frame}
\section{דוגמת ריצה}
\begin{frame}{ניתוח זמן ריצה}
\pause
\begin{itemize}[<+->]
	\item חלוקה לשני חלקים - זמן לינארי
	\item ספירת היפוכים עבור חצי מערך - \E{$T(\frac{n}{2})$}
	\item ספירת היפוכים בין שני החציים וסכימה יחד עם התוצאות של כל חצי -  זמן לינארי
\end{itemize}
\pause
סה"כ: 

$$T(n) = 2\cdot T(\frac{n}{2}) + O(n) = n log(n)$$
\end{frame}
\begin{frame}{קונבולציה}
\pause
\begin{itemize}[<+->]
	\item פעולה על שני וקטורים בדומה לחיבור, חיסור, כפל מטריצות או מכפלה 
	פנימית
	\item שימושי ב-
	\begin{itemize}[<+->]
		\item עיבוד אותות 
		\item מכפלה מהירה של מספרים טבעיים גדולים
		\item יצירת ממוצע נע (moving average)
	\end{itemize}
\end{itemize}
\end{frame}
\begin{frame}{קונבולוציה - בעצם אתם כבר מכירים את זה מכיתה ח'}
\pause
\begin{itemize}[<+->]
\item איך מכפילים שני פולינומים?
$$(4x^2 + 3x +1)\cdot (x^2 + x +2)=$$
\pause
$$4x^4 + 7x^3 +12x^2 + 7x+3$$
\item למעשה מה שעשיתם זה קונבולוציה למקדמי הפולינומים
\end{itemize}
\end{frame}
\begin{frame}{קונבולוציה - הגדרה}
\pause
\begin{itemize}[<+->]
	\item בהינתן שני וקטורים
	$$A=(a_0,a_1,\ldots,a_n)$$
		$$B=(b_0,b_1,\ldots,b_m)$$
	\item קונבולוציה של \E{$A$} עם \E{$B$} מסומנת ב: \E{$A*B = C$} 
	\item עבור וקטור 
	\E{$C$} באורך \E{$1+n+m$} המוגדר כך:
	$$c_i=\sum_{j,k:j+k=i}a_j\cdot b_k$$
\end{itemize}
\end{frame}
\begin{frame}{דוגמא}
\E{$(a_0,a_1,a_2)*(b_0,b_1,b_2)=(1,3,4)*(2,1,1) = (\onslide<3->{2},\onslide<5->{7},\onslide<7->{12},\onslide<8->{7},\onslide<8->{4})$ }
\pause
\begin{itemize}[<+->]
	\item \E{$c_0=a_0\cdot b_0=1\cdot 2$}
	\pause
	\item \E{$c_1=a_0\cdot b_1 +a_1\cdot b_0=1\cdot 1+3\cdot 2$}
		\pause
		\item \E{$c_2=a_0\cdot b_2 +a_1\cdot b_1 + a_2\cdot b_0=1\cdot 1+3 \cdot 1 + 4 \cdot 2 $}
\end{itemize}
\end{frame}
\begin{frame}{קונובולוציה - זה בעצם שקול להכפלת שני פולינומים}
\pause
\begin{itemize}[<+->]
	\item בדוגמא הקודמת:
	$$(1,3,4) \rightarrow 4x^2 + 3x +1$$ 
		$$(2,1,1) \rightarrow x^2 + x +2$$
	\item \E{$(1,3,4)*(2,1,1) \rightarrow (4x^2 + 3x +1)\cdot (x^2 + x +2)=$}
	$$4x^4 + 7x^3 +12x^2 + 7x+2 \rightarrow (2,7,12,7,4)$$
\end{itemize}

\end{frame}
\begin{frame}{הכפלת שני פולינומים}
\pause
\begin{itemize}[<+->]
	\item נגביל את הבעיה למקרה בו שני הפולינומים מאותו סדר, כלומר  \E{m=n} (אחרת פשוט אפשר להוסיף אפסים)
	\item מה זמן הריצה להכפלת שני פולינומים מסדר n?
	\item האם אפשר לעשות את זה בזמן טוב יותר?
\end{itemize}
\end{frame} 
\begin{frame}{ייצוג פולינומים}
בהינתן פולינום מדרגה \E{$n$} ניתן לייצג אותו:
\pause
\begin{itemize}[<+->]
	\item ע"י \E{$n+1$}  מקדמים 
	\item ע"י \E{$n+1$} נקודות
\end{itemize}
\pause
\textbf{המשפט היסודי של האלגברה:} דרך \E{$n+1$} נקודות עובר פולינום יחיד ממעלה \E{$n$}
\end{frame}
\begin{frame}{ההבדלים בין הייצוגים}
\pause
\begin{itemize}[<+->]
	\item קבלת הערך בנקודה
	\item מכפלת שני פולינומים
	\pause
	\begin{itemize}[<+->]
		\item בייצוג מקדמים - דורש זמן ריבועי
		\item בייצוג בנקודות - אם יש לנו \E{$2n+1$} נקודות של כל פולינום ניתן לחשב בזמן לינארי 
	\end{itemize}
\end{itemize}
\end{frame}
\begin{frame}{דוגמא}
\pause
\begin{itemize}[<+->]
	\item \E{$(1,1,1)*(1,0,2)=$}
	\item \E{$(x^2+x+1)\cdot (2x^2+1) =x^4+x^3+3x^2+2x+2= (2,2,3,1,1)$}
	\item 
	\item 
\end{itemize}
\end{frame}
\end{document}
