%! TEX program = lualatex
% find: (\\E{$((?!\$}).)+\$)
%replace: \E{\1}

%%%%%%%%%%
% BEAMER %
%%%%%%%%%%

\RequirePackage{luatex85}

\PassOptionsToPackage{luatex}{hyperref}

%\documentclass[luatex]{beamer}
\documentclass[handout]{beamer}

\usetheme{metropolis}

	\newcommand{\bigq}{\mathbb{Q}}
\newcommand{\bb}{\bfseries\large}
\newcommand{\class}{\mathcal}
\newcommand{\mbi}[1]{\textbf{\em{#1}}}
\newcommand{\bi}[1]{\textbf{\textit{#1}}}
\newcommand{\ita}[1]{\textit{#1}}
\newcommand{\p}{\cdot}
\newcommand{\req}{\emptyset}
\newcommand{\init}{\mathcal{J}}
\newcommand{\power}[1]{\class{P}(#1)}
\newcommand{\bigpower}[2]{\class{P}^{#1}(#2)}
\newcommand{\symetric}{\bigtriangleup}
\newcommand{\gx}[3]{\Gamma_{(#1,#2)}(#3)}
\newcommand{\bd}[3]{\Delta_{(#1,#2)}(#3)}
\newcommand{\bp}[3]{\Phi_{(#1,#2)}(#3)}
\newcommand{\eab}{\eta_{\alpha \beta}}
\newcommand{\sab}{\sigma_{\alpha\beta}}
\newcommand{\dab}{\class{D}_{\alpha\beta}}
\newcommand{\cab}{\class{C}_{\alpha\beta}}
\newcommand{\C}[1]{\class{#1}}
\newcommand{\ol}{\overline}
\newcommand{\into}{\longrightarrow}
\newcommand{\ifff}{\Longleftrightarrow}
\newcommand{\imply}{\Rightarrow}
\newcommand{\since}{\Longleftarrow}
\newcommand{\la}{\langle}
\newcommand{\ra}{\rangle}
\newcommand{\andd}{\wedge}
\newcommand{\orr}{\vee}
\newcommand{\ft}{(\C F \C T)_Q}
\newcommand{\pathopen}[3]{#1_{(\ol{#2},#3)}}
\newcommand{\fpi}[3]{#1_*(\Pi(#2,#3))}
\newcommand{\xpath}[2]{(\ol {#1},#2)}
\newcommand{\qn}[2]{\Lambda(#1,#2)}
\newcommand{\jh}{\class J \class H}
\newcommand{\bos}{\boldsymbol}
\newcommand{\f}{\mathbb}
\newcommand{\rss}{\la\psi_i:i\in \lambda\ra}
\newcommand{\ito}[1]{\overset{#1}{\longrightarrow}}
\newcommand{\pcomplex}[2]{\class #1(\class #2)}
\newcommand{\then}{\Rightarrow}
\newcommand{\ud}{{ {up-down-automaton}} }

\newcommand{\map}[1]{\underset{#1}{\mapsto}}
\newcommand{\mapi}[1]{\underset{#1}{\overset{i}{\mapsto}}}
\newcommand{\mapstar}[1]{\underset{#1}{\overset{*}{\mapsto}}}
\newcommand{\mapk}[2]{\underset{#1}{\overset{#2}{\mapsto}}}

\newcommand{\eql}{\equiv_L}
\newcommand{\der}[1]{\underset{#1}{\imply}}
\newcommand{\deri}[1]{\underset{#1}{\overset{i}{\imply}}}
\newcommand{\derstar}[1]{\underset{#1}{\overset{*}{\imply}}}
\newcommand{\derk}[2]{\underset{#1}{\overset{#2}{\imply}}}
\newcommand{\blank}{\bar{b}}
\newcommand{\R}[1]{#1}
\newcommand{\say}[1]{\color{blue}{#1}\color{black}}



% Fix the title separator
\makeatletter
\setbeamertemplate{title separator}{
	\bgroup
	\bodydir TLT
	\begin{tikzpicture}
	\fill[fg] (0,0) rectangle (\textwidth, \metropolis@titleseparator@linewidth);
	\end{tikzpicture}%
	\egroup
	\par%
}
\makeatother

% Fix the progress bar and make it progress from right to left
\makeatletter
\setbeamertemplate{progress bar in section page}{
	\setlength{\metropolis@progressonsectionpage}{%
		\textwidth * \ratio{\insertframenumber pt}{\inserttotalframenumber pt}%
	}%
	\bgroup
	\bodydir TLT
	\begin{tikzpicture}
	\fill[bg] (0,0) rectangle (\textwidth, \metropolis@progressonsectionpage@linewidth);
	\fill[fg] (\textwidth,0) rectangle (\textwidth-\metropolis@progressonsectionpage, \metropolis@progressonsectionpage@linewidth);
	\end{tikzpicture}%
	\egroup
}
\makeatother

% A more logical title page, juxtaposing `institute` to `author`. This modification is not necessary for Hebrew support.
\makeatletter
\setbeamertemplate{title page}{
	\begin{minipage}[b][\paperheight]{\textwidth}
		\ifx\inserttitlegraphic\@empty\else\usebeamertemplate*{title graphic}\fi
		\vfill%
		\ifx\inserttitle\@empty\else\usebeamertemplate*{title}\fi
		\ifx\insertsubtitle\@empty\else\usebeamertemplate*{subtitle}\fi
		\usebeamertemplate*{title separator}
		\ifx\beamer@shortauthor\@empty\else\usebeamertemplate*{author}\fi
		\ifx\insertinstitute\@empty\else\usebeamertemplate*{institute}\fi
		\ifx\insertdate\@empty\else\usebeamertemplate*{date}\fi
		\vfill
		\vspace*{1mm}
	\end{minipage}
}
\setbeamertemplate{author}{\vspace*{2em}\insertauthor\par\vspace*{0.25em}}
\setbeamertemplate{institute}{\insertinstitute\par}
\setbeamertemplate{date}{\vspace*{3mm}\insertdate\par}
\makeatother




%%%%%%%%%
% Babel %
%%%%%%%%%

\usepackage[nil,bidi=basic-r]{babel}
\babelprovide[import=he,main]{hebrew}
\babelprovide[import=en-GB]{english}

% For some reason Babel’s `\babelfont` doesn’t work
\setsansfont[Script=Hebrew]{Open Sans Hebrew}
\setmonofont{Fira Mono}
\renewcommand{\H}[1]{\foreignlanguage{hebrew}{\fontspec[Script=Hebrew]{Open Sans Hebrew}#1}}
\newcommand{\E}[1]{\foreignlanguage{english}{\fontspec{Open Sans}#1}}
\newcommand{\LR}[1]{{‏\textdir TLT #1}}




%%%%%%%%
% MISC %
%%%%%%%%

\usepackage{metalogo, fancyvrb}
\usepackage{cancel}
\newcommand{\smallurl}[1]{{\footnotesize\url{#1}}}



%%%%%%%%%%%%
% DOCUMENT %
%%%%%%%%%%%%

\begin{document}
	\title{ אלגוריתמים חמדניים - מסלולים קלים ביותר}
	\subtitle{מפגש 4}
	%\author{}
	%\institute{}
	\date{}
	
	
	\maketitle
\begin{frame}{שבוע שעבר ראינו}
\pause

\begin{itemize}[<+->]
	\item סיימנו סריקות בגרף (DFS,BFS)
	\item מיון טופולוגי
\end{itemize}
\end{frame}

\begin{frame}{התוכנית להיום}
\pause
\begin{itemize}[<+->]
	\item מסלולים קלים ביותר
\item אלגוריתם גנרי
\item אלגוריתם דייקסטרא
\end{itemize}
\end{frame}
\section{אלגוריתמים חמדניים}
\part{מסלולים קלים ביותר בגרפים מכוונים עם מקור יחיד}
\frame{\partpage}
\begin{frame}{הקדמה}
\pause
\begin{itemize}[<+->]
	\item נוסיף אספקט נוסף לגרף, גרף ממושקל.
	\item נוספת פונקציית משקל על הקשתות:
	$$w:E\rightarrow \mathbb{R}$$
	\item \textbf{הגדרה:} משקל של מסלול הינו סך משקולות הצלעות
	\item פונקציית מרחק $\delta$:
	$$\delta(s,v)=\delta(v) = \begin{cases}
	\infty, \quad \text{$v$- ל $s$ אין מסלול בין}\\
	\text{$v$-ל $s$-משקל מסלול קל ביותר מ },\quad \text{אחרת}
	\end{cases}$$
	\item \textbf{הערה: }במידה וקיים מעגל שלילי (=מעגל במשקל שלילי) במסלול כלשהו בין 
	\E{$s$}
	ל-
	\E{$v$}
	אזי הפונקציה
	$\delta$ 
	לא מוגדרת היטב. נשלים את ההגדרה במקרה זה כ-
	$$\delta(s,v)=-\infty$$
\end{itemize}
\end{frame}

\begin{frame}{ בעיית מציאת מסלולים קלים ביותר בגרפים מכוונים עם מקור יחיד ומשקולות אי-שליליים}
\begin{itemize}[<+->]
	\item \textbf{קלט:} 
	\E{$G=(V,E)$}
	מכוון,
	\E{$s\in V$}
	קודקוד מקור.
	\E{$w:E\rightarrow \mathbb{R}^+\cup\{0\}$}
	\item \textbf{יש למצוא:} לכל 
	\E{$v\in V$}
	מסלול קל ביותר מ-
	\E{$s$}
	ל-
		\E{$v$},
		אם קיים.
\end{itemize}
\end{frame}
\begin{frame}{הערות}
 \pause
\begin{itemize}[<+->]
	\item ראשית נחשב את המרחקים לכל $v\in V$ ולאחר מכן נשחזר את המסלולים עצמם ע"י שימוי במרחקים הנ"ל.
	\item אם $w(e)=c$ לכל קשת, איך נוכל לפתור את הבעיה?
	\item מסלול קל ביותר הינו בהכרח מסלול פשוט.
\end{itemize}
\end{frame}
\begin{frame}{טענה}

\pause
\begin{itemize}[<+->]
	\item תת מסלול של מסלול קל ביותר הינו מסלול קל ביותר. (הוכחה על הלוח)
	\item \textbf{מסקנה:} רישא של מסלול קל ביותר הינו גם מסלול קל ביותר.
\end{itemize}
\end{frame}
\begin{frame}{הערה נוספת}
\pause
\begin{itemize}[<+->]
\item לכל קשת \E{$(u,v)$} מתקיים:
$$\delta(s,v) \le \delta(s,u)+w(u,v)$$
\end{itemize}

\end{frame}
\begin{frame}{ אלגוריתם גנרי}
\LR{
	\fbox{
		\parbox{8cm}{
			\begin{itemize}
				\item[] \textbf{Init:} $\forall v\in V$, $d(v)\leftarrow \infty$,  $d(s)\leftarrow 0$ 
				\item[] \textbf{Step:} If exists $(u,v)$ s.t $d(v)>d(u)+w(u,v)$ do $Relax(u,v)$
			\end{itemize}
	}}
	
}
\pause
\LR{
	\fbox{
		\parbox{8cm}{
			{\sc Relax($u,v$) } 
			\begin{itemize}
				\item[] If $d(v)>d(u)+w(u,v)$
				\item[] $d(v)\leftarrow d(u)+w(u,v)$
			\end{itemize}
	}}
	
}
\end{frame}
\begin{frame}{טענה נשמרת}
בכל שלב באלגוריתם 
\E{$d(v)\ge\delta(s,v)$}
ואם
\E{ $d(v)<\infty$ }
 אזי 
\E{$d(v)$ }
משקל מסלול כלשהו מ-
\E{$s$} 
 ל-
 \E{$v$}
 .
\end{frame}
\section{הוכחה על הלוח}
\begin{frame}{משפט נכונות האלגוריתם הגנרי}
אם האלגוריתם עוצר אזי לכל 
\E{$v\in V$}
$$d(v)=\delta(s,v)$$
\end{frame}
\section{הוכחה על הלוח}
\section{דייקסטרא}
\end{document}
