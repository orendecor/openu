%! TEX program = lualatex
% find: (\$((?!\$).)+\$)
%replace: \E{\1}

%%%%%%%%%%
% BEAMER %
%%%%%%%%%%

\RequirePackage{luatex85}

\PassOptionsToPackage{luatex}{hyperref}

%\documentclass[luatex]{beamer}
\documentclass[handout]{beamer}

\usetheme{metropolis}

	\newcommand{\bigq}{\mathbb{Q}}
\newcommand{\bb}{\bfseries\large}
\newcommand{\class}{\mathcal}
\newcommand{\mbi}[1]{\textbf{\em{#1}}}
\newcommand{\bi}[1]{\textbf{\textit{#1}}}
\newcommand{\ita}[1]{\textit{#1}}
\newcommand{\p}{\cdot}
\newcommand{\req}{\emptyset}
\newcommand{\init}{\mathcal{J}}
\newcommand{\power}[1]{\class{P}(#1)}
\newcommand{\bigpower}[2]{\class{P}^{#1}(#2)}
\newcommand{\symetric}{\bigtriangleup}
\newcommand{\gx}[3]{\Gamma_{(#1,#2)}(#3)}
\newcommand{\bd}[3]{\Delta_{(#1,#2)}(#3)}
\newcommand{\bp}[3]{\Phi_{(#1,#2)}(#3)}
\newcommand{\eab}{\eta_{\alpha \beta}}
\newcommand{\sab}{\sigma_{\alpha\beta}}
\newcommand{\dab}{\class{D}_{\alpha\beta}}
\newcommand{\cab}{\class{C}_{\alpha\beta}}
\newcommand{\C}[1]{\class{#1}}
\newcommand{\ol}{\overline}
\newcommand{\into}{\longrightarrow}
\newcommand{\ifff}{\Longleftrightarrow}
\newcommand{\imply}{\Rightarrow}
\newcommand{\since}{\Longleftarrow}
\newcommand{\la}{\langle}
\newcommand{\ra}{\rangle}
\newcommand{\andd}{\wedge}
\newcommand{\orr}{\vee}
\newcommand{\ft}{(\C F \C T)_Q}
\newcommand{\pathopen}[3]{#1_{(\ol{#2},#3)}}
\newcommand{\fpi}[3]{#1_*(\Pi(#2,#3))}
\newcommand{\xpath}[2]{(\ol {#1},#2)}
\newcommand{\qn}[2]{\Lambda(#1,#2)}
\newcommand{\jh}{\class J \class H}
\newcommand{\bos}{\boldsymbol}
\newcommand{\f}{\mathbb}
\newcommand{\rss}{\la\psi_i:i\in \lambda\ra}
\newcommand{\ito}[1]{\overset{#1}{\longrightarrow}}
\newcommand{\pcomplex}[2]{\class #1(\class #2)}
\newcommand{\then}{\Rightarrow}
\newcommand{\ud}{{ {up-down-automaton}} }

\newcommand{\map}[1]{\underset{#1}{\mapsto}}
\newcommand{\mapi}[1]{\underset{#1}{\overset{i}{\mapsto}}}
\newcommand{\mapstar}[1]{\underset{#1}{\overset{*}{\mapsto}}}
\newcommand{\mapk}[2]{\underset{#1}{\overset{#2}{\mapsto}}}

\newcommand{\eql}{\equiv_L}
\newcommand{\der}[1]{\underset{#1}{\imply}}
\newcommand{\deri}[1]{\underset{#1}{\overset{i}{\imply}}}
\newcommand{\derstar}[1]{\underset{#1}{\overset{*}{\imply}}}
\newcommand{\derk}[2]{\underset{#1}{\overset{#2}{\imply}}}
\newcommand{\blank}{\bar{b}}
\newcommand{\R}[1]{#1}
\newcommand{\say}[1]{\color{blue}{#1}\color{black}}



% Fix the title separator
\makeatletter
\setbeamertemplate{title separator}{
	\bgroup
	\bodydir TLT
	\begin{tikzpicture}
	\fill[fg] (0,0) rectangle (\textwidth, \metropolis@titleseparator@linewidth);
	\end{tikzpicture}%
	\egroup
	\par%
}
\makeatother

% Fix the progress bar and make it progress from right to left
\makeatletter
\setbeamertemplate{progress bar in section page}{
	\setlength{\metropolis@progressonsectionpage}{%
		\textwidth * \ratio{\insertframenumber pt}{\inserttotalframenumber pt}%
	}%
	\bgroup
	\bodydir TLT
	\begin{tikzpicture}
	\fill[bg] (0,0) rectangle (\textwidth, \metropolis@progressonsectionpage@linewidth);
	\fill[fg] (\textwidth,0) rectangle (\textwidth-\metropolis@progressonsectionpage, \metropolis@progressonsectionpage@linewidth);
	\end{tikzpicture}%
	\egroup
}
\makeatother

% A more logical title page, juxtaposing `institute` to `author`. This modification is not necessary for Hebrew support.
\makeatletter
\setbeamertemplate{title page}{
	\begin{minipage}[b][\paperheight]{\textwidth}
		\ifx\inserttitlegraphic\@empty\else\usebeamertemplate*{title graphic}\fi
		\vfill%
		\ifx\inserttitle\@empty\else\usebeamertemplate*{title}\fi
		\ifx\insertsubtitle\@empty\else\usebeamertemplate*{subtitle}\fi
		\usebeamertemplate*{title separator}
		\ifx\beamer@shortauthor\@empty\else\usebeamertemplate*{author}\fi
		\ifx\insertinstitute\@empty\else\usebeamertemplate*{institute}\fi
		\ifx\insertdate\@empty\else\usebeamertemplate*{date}\fi
		\vfill
		\vspace*{1mm}
	\end{minipage}
}
\setbeamertemplate{author}{\vspace*{2em}\insertauthor\par\vspace*{0.25em}}
\setbeamertemplate{institute}{\insertinstitute\par}
\setbeamertemplate{date}{\vspace*{3mm}\insertdate\par}
\makeatother




%%%%%%%%%
% Babel %
%%%%%%%%%

\usepackage[nil,bidi=basic-r]{babel}
\babelprovide[import=he,main]{hebrew}
\babelprovide[import=en-GB]{english}

% For some reason Babel’s `\babelfont` doesn’t work
\setsansfont[Script=Hebrew]{Open Sans Hebrew}
\setmonofont{Fira Mono}
\renewcommand{\H}[1]{\foreignlanguage{hebrew}{\fontspec[Script=Hebrew]{Open Sans Hebrew}#1}}
\newcommand{\E}[1]{\foreignlanguage{english}{\fontspec{Open Sans}#1}}
\newcommand{\LR}[1]{{‏\textdir TLT #1}}




%%%%%%%%
% MISC %
%%%%%%%%

\usepackage{metalogo, fancyvrb}
\usepackage{cancel}
\newcommand{\smallurl}[1]{{\footnotesize\url{#1}}}



%%%%%%%%%%%%
% DOCUMENT %
%%%%%%%%%%%%

\begin{document}
	\title{בעיית העץ הפורש המינימאלי (MST)}
	\subtitle{תרגול 3}
	%\author{}
	%\institute{}
	\date{}
	
	
	\frame{\titlepage}
\begin{frame}
\begin{figure}
	\centering
	\includegraphics[width=\linewidth]{imgs/island}
\end{figure}


\end{frame}
\begin{frame}
\begin{figure}
	\centering
	\includegraphics[width=.6\linewidth]{imgs/mst}
\end{figure}

\end{frame}
\part{הגדרות ומשפטים}
\frame{\partpage}
\begin{frame}{הגדרות מתורת הגרפים}
\begin{itemize}[<+->]
	\item \textbf{גרף פשוט} 
	גרף \E{$G$} הוא פשוט אם אין בו צלעות עצמיות (צלע מקודקוד לעצמו) או צלעות כפולות (יותר מצלע אחת בין זוג קודקודים).
	\item \textbf{מסלול}   מסלול בגרף  
	\E{$G=(V,E)$}
	הוא סידרת קודקודים 
	\E{$<v_0,v_1,\ldots,v_k>$}
	, כאשר לכל 
	\E{$0\le i < k$}
	מתקיים 
	\E{$(v_i,v_{i+1})\in E$}.
	\item \textbf{מסלול פשוט}  מסלול  \E{$<v_0,v_1,\ldots,v_k>$}, הוא פשוט אם כל קדקודיו שונים זמ"ז, כלומר לכל   
	\E{$0\le i < j \le k$}
	מתקיים 
	\E{$v_i\ne v_j$}
	.
	\item \textbf{מעגל}   מסלול 
	\E{$<v_0,v_1,\ldots,v_k>$}
	, כך ש- 
	\E{$v_0= v_k$}
	.
\end{itemize}
\end{frame}
\begin{frame}{עץ פורש - הגדרה}
 \textbf{עץ פורש }  עץ פורש של גרף קשיר 
\E{$G=(V,E)$}
הוא תת-גרף   
\E{$T=(V,E_T)$}
של \E{$G$} \pause שמכיל את כל קדקודי \E{$G$} וחלק מצלעותיו (כלומר \E{$E_T\subseteq E$}), והוא מקיים:
\pause 
\begin{enumerate}[<+->]
	\item \E{$T$} חסר מעגלים.
	\item \E{$T$} קשיר (כלומר לכל שני קודקודים קיים מסלול המקשר ביניהם).
\end{enumerate}
\end{frame}
\begin{frame}{משפט 1 (הוכח בכיתה)}
יהי \E{$H$} גרף (פשוט ולא מכוון). התנאים הבאים שקולים זה לזה:
\pause 
 \begin{enumerate}[<+->]
 	\item 	\E{$H$} קשיר וחסר מעגלים,

 	\item \E{$H$} חסר מעגלים ו- \E{$|E| = |V|-1$},

 	\item \E{$H$} קשיר ו-\E{$|E| = |V|-1$}
 	\item יש ב-\E{$H$} מסלול פשוט יחיד בין כל זוג צמתים.

 \end{enumerate}
\pause 
כדי להוכיח שתת הגרף   \E{$T=(V,E_T)$} של הגרף  \E{$G=(V,E)$} הוא עץ פורש של  \E{$G$}, מספיק להראות שאחד מהתנאים הנ"ל מתקיים.
\end{frame}
\begin{frame}{משפט 2 (ללא הוכחה)}
יהי 
\begin{itemize}[<+->]
	\item \E{$G$} גרף, 
	\item \E{$T=(V,E_T)$}
	עץ פורש של 
	\E{$G$}
	\item ו- 
	\E{$ e\in E\setminus E_T$}
	.
\end{itemize}
\pause 
    הגרף 
    \E{$H=(V,E_T\cup \{e\})$}
    מכיל מעגל \pause  ולכל צלע  
    \E{$e'\in E_T$}
     במעגל, \\ \pause 
     הגרף 
         \E{$T'=(V,(E_T\cup \{e\})\setminus \{e'\})$}
       הוא עץ פורש של \E{$G$}.
\end{frame}
\begin{frame}{הגדרות משקול של עץ פורש}
\begin{itemize}[<+->]
	\item \textbf{עלות של עץ פורש } 	 בהינתן גרף  \E{$G=(V,E)$} ופונקצית משקל  \E{$w:E\rightarrow \mathbb{R}$}, נגדיר \textbf{עלות} של עץ פורש \E{$T$} להיות סך המשקלות של כל צלעותיו, כלומר 
	 \E{$w(T) = \sum_{e\in T} w(e)$}
	  .

	\item \textbf{עץ פורש מינימאלי (MST)} של גרף  \E{$G$} הוא עץ פורש שעלותו מינימאלית מבין כל עלויות העצים הפורשים את  \E{$G$}, כלומר נבקש למצוא עץ פורש \E{$T$} כך ש : 
	$$w(T) = min_{\tilde T\text{ is a spanning tree}}\{w(\tilde T)\}$$
	 .
	\item \textbf{הערה-} עץ פורש מינימאלי אינו בהכרח יחיד. יתכנו כמה כאלו, אבל לכולם, כמובן, אותו משקל.

\end{itemize}
\end{frame}
\part{שאלות}
\frame{\partpage}
\begin{frame}{שאלות}
לכל השאלות בתרגול נתון לנו גרף לא מכוון וקשיר \E{$G=(V,E)$} ופונקצית משקל על קשתות הגרף  \E{$w:E\rightarrow \mathbb{R}$}.
\pause \\
\textbf{שאלה 1:
}

יהי  \E{$T=(V,E_T)$}
עץ פורש של 
\E{$G$} עץ פורש מינימאלי של  
\E{$G$}
תחת פונקצית המשקל  
\E{$w$}
. \\
\pause 
נגדיר את 
\E{$w'$}
  להיות פונקצית משקל באופן הבא:
\\
  \pause 
  
  \E{$w'(e)=w(e)+c$}
, כאשר 
\E{$c\in \mathbb{R}$}
 קבוע כלשהו. \\
 \pause 
 האם  \E{$T$} הוא עץ פורש מינימאלי של  \E{$G$} תחת פונקצית המשקל \E{$w'$}?

\end{frame}
\end{document}
