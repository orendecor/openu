%! TEX program = lualatex
% find: (\\E{$((?!\$}).)+\$)
%replace: \E{\1}

%%%%%%%%%%
% BEAMER %
%%%%%%%%%%

\RequirePackage{luatex85}

\PassOptionsToPackage{luatex}{hyperref}

%\documentclass[luatex]{beamer}
\documentclass[handout]{beamer}

\usetheme{metropolis}

	\newcommand{\bigq}{\mathbb{Q}}
\newcommand{\bb}{\bfseries\large}
\newcommand{\class}{\mathcal}
\newcommand{\mbi}[1]{\textbf{\em{#1}}}
\newcommand{\bi}[1]{\textbf{\textit{#1}}}
\newcommand{\ita}[1]{\textit{#1}}
\newcommand{\p}{\cdot}
\newcommand{\req}{\emptyset}
\newcommand{\init}{\mathcal{J}}
\newcommand{\power}[1]{\class{P}(#1)}
\newcommand{\bigpower}[2]{\class{P}^{#1}(#2)}
\newcommand{\symetric}{\bigtriangleup}
\newcommand{\gx}[3]{\Gamma_{(#1,#2)}(#3)}
\newcommand{\bd}[3]{\Delta_{(#1,#2)}(#3)}
\newcommand{\bp}[3]{\Phi_{(#1,#2)}(#3)}
\newcommand{\eab}{\eta_{\alpha \beta}}
\newcommand{\sab}{\sigma_{\alpha\beta}}
\newcommand{\dab}{\class{D}_{\alpha\beta}}
\newcommand{\cab}{\class{C}_{\alpha\beta}}
\newcommand{\C}[1]{\class{#1}}
\newcommand{\ol}{\overline}
\newcommand{\into}{\longrightarrow}
\newcommand{\ifff}{\Longleftrightarrow}
\newcommand{\imply}{\Rightarrow}
\newcommand{\since}{\Longleftarrow}
\newcommand{\la}{\langle}
\newcommand{\ra}{\rangle}
\newcommand{\andd}{\wedge}
\newcommand{\orr}{\vee}
\newcommand{\ft}{(\C F \C T)_Q}
\newcommand{\pathopen}[3]{#1_{(\ol{#2},#3)}}
\newcommand{\fpi}[3]{#1_*(\Pi(#2,#3))}
\newcommand{\xpath}[2]{(\ol {#1},#2)}
\newcommand{\qn}[2]{\Lambda(#1,#2)}
\newcommand{\jh}{\class J \class H}
\newcommand{\bos}{\boldsymbol}
\newcommand{\f}{\mathbb}
\newcommand{\rss}{\la\psi_i:i\in \lambda\ra}
\newcommand{\ito}[1]{\overset{#1}{\longrightarrow}}
\newcommand{\pcomplex}[2]{\class #1(\class #2)}
\newcommand{\then}{\Rightarrow}
\newcommand{\ud}{{ {up-down-automaton}} }

\newcommand{\map}[1]{\underset{#1}{\mapsto}}
\newcommand{\mapi}[1]{\underset{#1}{\overset{i}{\mapsto}}}
\newcommand{\mapstar}[1]{\underset{#1}{\overset{*}{\mapsto}}}
\newcommand{\mapk}[2]{\underset{#1}{\overset{#2}{\mapsto}}}

\newcommand{\eql}{\equiv_L}
\newcommand{\der}[1]{\underset{#1}{\imply}}
\newcommand{\deri}[1]{\underset{#1}{\overset{i}{\imply}}}
\newcommand{\derstar}[1]{\underset{#1}{\overset{*}{\imply}}}
\newcommand{\derk}[2]{\underset{#1}{\overset{#2}{\imply}}}
\newcommand{\blank}{\bar{b}}
\newcommand{\R}[1]{#1}
\newcommand{\say}[1]{\color{blue}{#1}\color{black}}



% Fix the title separator
\makeatletter
\setbeamertemplate{title separator}{
	\bgroup
	\bodydir TLT
	\begin{tikzpicture}
	\fill[fg] (0,0) rectangle (\textwidth, \metropolis@titleseparator@linewidth);
	\end{tikzpicture}%
	\egroup
	\par%
}
\makeatother

% Fix the progress bar and make it progress from right to left
\makeatletter
\setbeamertemplate{progress bar in section page}{
	\setlength{\metropolis@progressonsectionpage}{%
		\textwidth * \ratio{\insertframenumber pt}{\inserttotalframenumber pt}%
	}%
	\bgroup
	\bodydir TLT
	\begin{tikzpicture}
	\fill[bg] (0,0) rectangle (\textwidth, \metropolis@progressonsectionpage@linewidth);
	\fill[fg] (\textwidth,0) rectangle (\textwidth-\metropolis@progressonsectionpage, \metropolis@progressonsectionpage@linewidth);
	\end{tikzpicture}%
	\egroup
}
\makeatother

% A more logical title page, juxtaposing `institute` to `author`. This modification is not necessary for Hebrew support.
\makeatletter
\setbeamertemplate{title page}{
	\begin{minipage}[b][\paperheight]{\textwidth}
		\ifx\inserttitlegraphic\@empty\else\usebeamertemplate*{title graphic}\fi
		\vfill%
		\ifx\inserttitle\@empty\else\usebeamertemplate*{title}\fi
		\ifx\insertsubtitle\@empty\else\usebeamertemplate*{subtitle}\fi
		\usebeamertemplate*{title separator}
		\ifx\beamer@shortauthor\@empty\else\usebeamertemplate*{author}\fi
		\ifx\insertinstitute\@empty\else\usebeamertemplate*{institute}\fi
		\ifx\insertdate\@empty\else\usebeamertemplate*{date}\fi
		\vfill
		\vspace*{1mm}
	\end{minipage}
}
\setbeamertemplate{author}{\vspace*{2em}\insertauthor\par\vspace*{0.25em}}
\setbeamertemplate{institute}{\insertinstitute\par}
\setbeamertemplate{date}{\vspace*{3mm}\insertdate\par}
\makeatother




%%%%%%%%%
% Babel %
%%%%%%%%%

\usepackage[nil,bidi=basic-r]{babel}
\babelprovide[import=he,main]{hebrew}
\babelprovide[import=en-GB]{english}

% For some reason Babel’s `\babelfont` doesn’t work
\setsansfont[Script=Hebrew]{Open Sans Hebrew}
\setmonofont{Fira Mono}
\renewcommand{\H}[1]{\foreignlanguage{hebrew}{\fontspec[Script=Hebrew]{Open Sans Hebrew}#1}}
\newcommand{\E}[1]{\foreignlanguage{english}{\fontspec{Open Sans}#1}}
\newcommand{\LR}[1]{{‏\textdir TLT #1}}




%%%%%%%%
% MISC %
%%%%%%%%

\usepackage{metalogo, fancyvrb}
\usepackage{cancel}
\newcommand{\smallurl}[1]{{\footnotesize\url{#1}}}



%%%%%%%%%%%%
% DOCUMENT %
%%%%%%%%%%%%

\begin{document}
	\title{ סריקות לעומק ומיון טופולוגי}
	\subtitle{מפגש 3}
	%\author{}
	%\institute{}
	\date{}
	
	
	\maketitle
\begin{frame}{שבוע שעבר ראינו}
\pause

\begin{itemize}[<+->]
	\item מעבר על BFS
	\item תכנון אלגוריתם מבוסס רדוקציה
\end{itemize}
\end{frame}

\begin{frame}{התוכנית להיום}
\pause
\begin{itemize}[<+->]
	\item סריקת עומק - DFS 
	\item מיון טופולוגי
\end{itemize}
\end{frame}

\part{DFS - סריקה לעומק}
\frame{\partpage}
\begin{frame}{הקדמה}
האלגוריתם DFS פעול באופן רקורסיבי כאשר תמיד יעדיף לחפש "עמוק" יותר בגרף לפני מעבר לשכנים נוספים של קודקוד נסרק.
\pause
\begin{itemize}[<+->]
	\item בסריקה נוצר יער עומר \E{$G_\pi$}.
	צלע ביער מהסוג
	\E{$(\pi(x),x)$}.
	\item לכל קודקוד 3 מצבים אפשריים:
	\begin{itemize}[<+->]
		\item קודקוד מאותחל בלבן - ונשאר לבן כל עוד לא התגלה בסריקה
		\item קודקוד הפוך אפור ברגע גילויו
		\item קודקוד הופל מאפור לשחור לאחר שהסתיימה סריקת כל הקודקודים שנגישים ממנו.
	\end{itemize}
\end{itemize}
\end{frame}

\begin{frame}{הקדמה}
\begin{itemize}[<+->]
	\item :לכל קודקוד מספר שדות
	\begin{itemize}[<+->]
\item 
\E{$d(v)$} - זמן גילוי הקודקוד
\item \E{$f(v)$} - זמן סיום הטיפול הקודקוד
\item \E{$\pi(v)$} - קודקוד אשר ממנו התגלה 
\E{$v$}.
\item \E{$color(v)$} - צבע הקודקוד
	\end{itemize}
\end{itemize}
\end{frame}
\section{דוגמת ריצה על הלוח}
\begin{frame}{DFS - האלגוריתם}
\LR{
	\fbox{
		\parbox{8cm}{
			{\sc DFS($G$) } 
			\begin{itemize}
				\item[] $Init: t=1\quad$ color[$u$]=white for all $u\in V$\\
				\item[] FOR $u\in V$ DO
				\begin{itemize}
					\item[] IF color[$u$] = white THEN
					\begin{itemize}
						\item[] {\sc DFS-Visit(u)}
					\end{itemize}
				\end{itemize}
			\end{itemize}
	}}
	
}
\end{frame}
\begin{frame}{DFS - האלגוריתם}
\LR{
	\fbox{
		\parbox{8cm}{
			{\sc DFS-Visit($u$) } 
			\begin{itemize}
				\item[] color[$u$] = gray
				\item[] $d[u]$ = t
				\item[] t = t + 1
				\item[] FOR $(u,v)\in E$ DO
				\begin{itemize}
					\item[] IF color[$v$] = white THEN
					\begin{itemize}
						\item[] $\pi[v] = u$
						\item[] {\sc DFS-Visit(v)}
					\end{itemize}
				\end{itemize}
				\item[] color[$u$] = black
				\item[] $f[u]$ = t
				\item[] t = t + 1
			\end{itemize}
	}}
	
}
\end{frame}
\begin{frame}{תכונות DFS}
\pause
\begin{itemize}[<+->]
	\item סורק את כל הקודקודים
	\item מוצא מעגלים

	\item מציאת מיון טופולוגי
	\item סיבוכיות \E{$O(|V|+|E|)$}
\end{itemize}
\end{frame}

\begin{frame}{שאלה 3}
\textbf{הוכיחו או הפריכו את הטענה הבאה:}
יהי גרף קשיר ולא מכוון ויהי
\E{$u\in V$}
. אם קיימת הרצת
\E{$DFS$} 
 מ-
 \E{$u$}
  על
   \E{$G$}
    והרצת
     \E{$BFS$}
      מ-
       \E{$u$}
       על
        \E{$G$}
         הנותנות את אותו עץ
          \E{$T$}
           אז בהכרח
            \E{$G=T$}
\end{frame}
\begin{frame}{מיון טופולוגי}
\pause
\begin{itemize}[<+->]
	\item  מיון טופולוגי של גרף מכוון
\E{$G=(V,E)$}
 הינו סידור

\E{$ (v_1,...,v_n)$}
   של קודקודי הגרף, כך שלכל

\E{$1 \le i,j \le n$}
אם
\E{$ i < j$}
  אז אין קשתות מ
   \E{$j$}
    ל
     \E{$i$}
      בגרף
\end{itemize}
\pause
\begin{center}
\includegraphics[width=0.7\linewidth]{imgs/topologyorder}
\end{center}
\end{frame}
\section{דוגמה על הלוח}
\begin{frame}{מיון טופולוגי}
\textbf{משפט:} אם הגרף DAG אזי יש מיון טופולוגי.
\\
הוכחה בבניה (עמוד 111 בספר).
\end{frame}
\begin{frame}{מיון טופולוגי באמצעות DFS}
\pause
\begin{itemize}[<+->]
	\item \textbf{טענה: } אם 
	\E{$G$}
	חסר מעגלים ו-
	\E{$(u,v)\in E$}
	אז בכל סריקת 
	\E{DFS(G)}
	מתקיים
	\E{$f(v)<f(u)$}
	\item\textbf{ אלגוריתם מבוסס 
		\E{DFS}
				 למציאת מיון טופולוגי:
	 }

\begin{itemize}[<+->]
	\item הרץ 
	\E{DFS(G)}
	\item (אם לא התגלתה קשת אחורה) החזר סידור של 
	\E{$V$}
	לפי זמני 
	\E{$f$}
	בסדר יורד.
\end{itemize}
\pause
נכונות האלגוריתם נובעת מנכונות הטענה (על הלוח).
\end{itemize}
\end{frame}


\begin{frame}{שאלה 4}
 \pause
 \begin{itemize}[<+->]
 	\item \textbf{הגדרה:} גרף מעורב הוא גרף שבו כמה מהקשתות
מכוונות והאחרות אינן מכוונות. 
 	\item הוכיחו שאם בגרף מעורב התת-גרף המתקבל מכל
צמתי הגרף ומהקשתות המכוונות בלבד אינו מכיל  מעגל מכוון, אז תמיד ניתן לכוון את הקשתות הלא
מכוונות כך שבגרף המכוון המתקבל אין מעגל מכוון. הראו כיצד ניתן למצוא כיוון מתאים לקשתות בזמן לינארי.
 \end{itemize}
\end{frame}
\begin{frame}{שאלה 5}
נתון גרף מכוון.\\
כיצד יכול להשתנות מספר הרכיבים
 הקשירים היטב בגרף אם מוסיפים קשת חדשה?
\end{frame}
\begin{frame}{שאלה 6}
\textbf{הוכח או הפרך-}\\
אם בגרף מכוון יש קשתות הנכנסות לצומת u וגם קשתות היוצאות ממנו, אזי לא ייתכן שבהרצת DFS על הגרף הצומת u יימצא בעץ המכיל אותו בלבד.
\end{frame}

\begin{frame}{שאלה 7}
\textbf{הוכח או הפרך-}\\
יהי גרף קשיר ולא מכוון, יהי
 \E{$s\in V$}
  ויהי
   \E{$T$}
    עץ
המתקבל מהרצת
 \E{DFS}
  על
   
  \E{$G$}
   מ-
  \E{$s$}
   . אז עומקו של
    \E{$T$}
      הוא לפחות כעומקו של כל עץ המתקבל מהרצת 
      \E{BFS}
      על
      \E{$G$}
      מ-
      \E{$s$}.
\end{frame}

\begin{frame}{שאלה 8}
יהי 
\E{$G$}
גרף לא מכוון וקשיר. הוכח או הפרך:
\pause
\begin{itemize}[<+->]
	\item כל עץ המתקבל מריצת 
	\E{DFS}
	 על
	 \E{$G$} 
	 ניתן
לקבל על ידי ריצת
 \E{BFS}
  על
    \E{$G$} 
   . 
   \item כל עץ המתקבל מריצת 
   \E{BFS}
   על
   \E{$G$} 
   ניתן
   לקבל על ידי ריצת
   \E{DFS}
   על
   \E{$G$} 
   . 
\end{itemize}
\end{frame}
\end{document}
