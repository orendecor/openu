%! TEX program = lualatex
% find: (\\E{$((?!\$}).)+\$)
%replace: \E{\1}

%%%%%%%%%%
% BEAMER %
%%%%%%%%%%

\RequirePackage{luatex85}

\PassOptionsToPackage{luatex}{hyperref}

%\documentclass[luatex]{beamer}
\documentclass[handout]{beamer}

\usetheme{metropolis}

	\newcommand{\bigq}{\mathbb{Q}}
\newcommand{\bb}{\bfseries\large}
\newcommand{\class}{\mathcal}
\newcommand{\mbi}[1]{\textbf{\em{#1}}}
\newcommand{\bi}[1]{\textbf{\textit{#1}}}
\newcommand{\ita}[1]{\textit{#1}}
\newcommand{\p}{\cdot}
\newcommand{\req}{\emptyset}
\newcommand{\init}{\mathcal{J}}
\newcommand{\power}[1]{\class{P}(#1)}
\newcommand{\bigpower}[2]{\class{P}^{#1}(#2)}
\newcommand{\symetric}{\bigtriangleup}
\newcommand{\gx}[3]{\Gamma_{(#1,#2)}(#3)}
\newcommand{\bd}[3]{\Delta_{(#1,#2)}(#3)}
\newcommand{\bp}[3]{\Phi_{(#1,#2)}(#3)}
\newcommand{\eab}{\eta_{\alpha \beta}}
\newcommand{\sab}{\sigma_{\alpha\beta}}
\newcommand{\dab}{\class{D}_{\alpha\beta}}
\newcommand{\cab}{\class{C}_{\alpha\beta}}
\newcommand{\C}[1]{\class{#1}}
\newcommand{\ol}{\overline}
\newcommand{\into}{\longrightarrow}
\newcommand{\ifff}{\Longleftrightarrow}
\newcommand{\imply}{\Rightarrow}
\newcommand{\since}{\Longleftarrow}
\newcommand{\la}{\langle}
\newcommand{\ra}{\rangle}
\newcommand{\andd}{\wedge}
\newcommand{\orr}{\vee}
\newcommand{\ft}{(\C F \C T)_Q}
\newcommand{\pathopen}[3]{#1_{(\ol{#2},#3)}}
\newcommand{\fpi}[3]{#1_*(\Pi(#2,#3))}
\newcommand{\xpath}[2]{(\ol {#1},#2)}
\newcommand{\qn}[2]{\Lambda(#1,#2)}
\newcommand{\jh}{\class J \class H}
\newcommand{\bos}{\boldsymbol}
\newcommand{\f}{\mathbb}
\newcommand{\rss}{\la\psi_i:i\in \lambda\ra}
\newcommand{\ito}[1]{\overset{#1}{\longrightarrow}}
\newcommand{\pcomplex}[2]{\class #1(\class #2)}
\newcommand{\then}{\Rightarrow}
\newcommand{\ud}{{ {up-down-automaton}} }

\newcommand{\map}[1]{\underset{#1}{\mapsto}}
\newcommand{\mapi}[1]{\underset{#1}{\overset{i}{\mapsto}}}
\newcommand{\mapstar}[1]{\underset{#1}{\overset{*}{\mapsto}}}
\newcommand{\mapk}[2]{\underset{#1}{\overset{#2}{\mapsto}}}

\newcommand{\eql}{\equiv_L}
\newcommand{\der}[1]{\underset{#1}{\imply}}
\newcommand{\deri}[1]{\underset{#1}{\overset{i}{\imply}}}
\newcommand{\derstar}[1]{\underset{#1}{\overset{*}{\imply}}}
\newcommand{\derk}[2]{\underset{#1}{\overset{#2}{\imply}}}
\newcommand{\blank}{\bar{b}}
\newcommand{\R}[1]{#1}
\newcommand{\say}[1]{\color{blue}{#1}\color{black}}



% Fix the title separator
\makeatletter
\setbeamertemplate{title separator}{
	\bgroup
	\bodydir TLT
	\begin{tikzpicture}
	\fill[fg] (0,0) rectangle (\textwidth, \metropolis@titleseparator@linewidth);
	\end{tikzpicture}%
	\egroup
	\par%
}
\makeatother

% Fix the progress bar and make it progress from right to left
\makeatletter
\setbeamertemplate{progress bar in section page}{
	\setlength{\metropolis@progressonsectionpage}{%
		\textwidth * \ratio{\insertframenumber pt}{\inserttotalframenumber pt}%
	}%
	\bgroup
	\bodydir TLT
	\begin{tikzpicture}
	\fill[bg] (0,0) rectangle (\textwidth, \metropolis@progressonsectionpage@linewidth);
	\fill[fg] (\textwidth,0) rectangle (\textwidth-\metropolis@progressonsectionpage, \metropolis@progressonsectionpage@linewidth);
	\end{tikzpicture}%
	\egroup
}
\makeatother

% A more logical title page, juxtaposing `institute` to `author`. This modification is not necessary for Hebrew support.
\makeatletter
\setbeamertemplate{title page}{
	\begin{minipage}[b][\paperheight]{\textwidth}
		\ifx\inserttitlegraphic\@empty\else\usebeamertemplate*{title graphic}\fi
		\vfill%
		\ifx\inserttitle\@empty\else\usebeamertemplate*{title}\fi
		\ifx\insertsubtitle\@empty\else\usebeamertemplate*{subtitle}\fi
		\usebeamertemplate*{title separator}
		\ifx\beamer@shortauthor\@empty\else\usebeamertemplate*{author}\fi
		\ifx\insertinstitute\@empty\else\usebeamertemplate*{institute}\fi
		\ifx\insertdate\@empty\else\usebeamertemplate*{date}\fi
		\vfill
		\vspace*{1mm}
	\end{minipage}
}
\setbeamertemplate{author}{\vspace*{2em}\insertauthor\par\vspace*{0.25em}}
\setbeamertemplate{institute}{\insertinstitute\par}
\setbeamertemplate{date}{\vspace*{3mm}\insertdate\par}
\makeatother




%%%%%%%%%
% Babel %
%%%%%%%%%

\usepackage[nil,bidi=basic-r]{babel}
\babelprovide[import=he,main]{hebrew}
\babelprovide[import=en-GB]{english}

% For some reason Babel’s `\babelfont` doesn’t work
\setsansfont[Script=Hebrew]{Open Sans Hebrew}
\setmonofont{Fira Mono}
\renewcommand{\H}[1]{\foreignlanguage{hebrew}{\fontspec[Script=Hebrew]{Open Sans Hebrew}#1}}
\newcommand{\E}[1]{\foreignlanguage{english}{\fontspec{Open Sans}#1}}
\newcommand{\LR}[1]{{‏\textdir TLT #1}}




%%%%%%%%
% MISC %
%%%%%%%%

\usepackage{metalogo, fancyvrb}
\usepackage{cancel}
\newcommand{\smallurl}[1]{{\footnotesize\url{#1}}}



%%%%%%%%%%%%
% DOCUMENT %
%%%%%%%%%%%%

\begin{document}
	\title{ סריקות לרוחב ורדוקציות}
	\subtitle{מפגש 2}
	%\author{}
	%\institute{}
	\date{}
	
	
	\maketitle
\begin{frame}{שבוע שעבר ראינו}
\pause
\begin{itemize}[<+->]
	\item מבוא והגדרות בסיסיות 
	\item חזרה מהירה על ניתוח זמני ריצה
	\item מושגים בסיסיים מעולם הגרפים
\end{itemize}
\end{frame}

\begin{frame}{התוכנית להיום}
\pause
\begin{itemize}[<+->]
	\item מעבר על BFS
	\item תכנון אלגוריתם מבוסס רדוקציה
	\item  DFS  -כמה שהזמן יאפשר לנו 
\end{itemize}
\end{frame}
\part{BFS - סריקה לרוחב}
\frame{\partpage}
\begin{frame}
\vspace*{-1pt}
\makebox[\linewidth]{\includegraphics[page=34,width=\paperwidth]{pdfs/lec1}}
\end{frame}
\begin{frame}
\vspace*{-1pt}
\makebox[\linewidth]{\includegraphics[page=35,width=\paperwidth]{pdfs/lec1}}
\end{frame}
\begin{frame}
\vspace*{-1pt}
\makebox[\linewidth]{\includegraphics[page=36,width=\paperwidth]{pdfs/lec1}}
\end{frame}
\begin{frame}
\vspace*{-1pt}
\makebox[\linewidth]{\includegraphics[page=37,width=\paperwidth]{pdfs/lec1}}
\end{frame}
\begin{frame}
\vspace*{-1pt}
\makebox[\linewidth]{\includegraphics[page=38,width=\paperwidth]{pdfs/lec1}}
\end{frame}
\begin{frame}
\vspace*{-1pt}
\makebox[\linewidth]{\includegraphics[page=39,width=\paperwidth]{pdfs/lec1}}
\end{frame}
\begin{frame}
\vspace*{-1pt}
\makebox[\linewidth]{\includegraphics[page=40,width=\paperwidth]{pdfs/lec1}}
\end{frame}
\begin{frame}
\vspace*{-1pt}
\makebox[\linewidth]{\includegraphics[page=41,width=\paperwidth]{pdfs/lec1}}
\end{frame}
\begin{frame}
\vspace*{-1pt}
\makebox[\linewidth]{\includegraphics[page=42,width=\paperwidth]{pdfs/lec1}}
\end{frame}
\begin{frame}
\vspace*{-1pt}
\makebox[\linewidth]{\includegraphics[page=43,width=\paperwidth]{pdfs/lec1}}
\end{frame}
\begin{frame}
\vspace*{-1pt}
\makebox[\linewidth]{\includegraphics[page=44,width=\paperwidth]{pdfs/lec1}}
\end{frame}
\begin{frame}
\vspace*{-1pt}
\makebox[\linewidth]{\includegraphics[page=45,width=\paperwidth]{pdfs/lec1}}
\end{frame}
\begin{frame}
\vspace*{-1pt}
\makebox[\linewidth]{\includegraphics[page=46,width=\paperwidth]{pdfs/lec1}}
\end{frame}
\begin{frame}
\vspace*{-1pt}
\makebox[\linewidth]{\includegraphics[page=47,width=\paperwidth]{pdfs/lec1}}
\end{frame}
\begin{frame}
\vspace*{-1pt}
\makebox[\linewidth]{\includegraphics[page=48,width=\paperwidth]{pdfs/lec1}}
\end{frame}
\begin{frame}
\vspace*{-1pt}
\makebox[\linewidth]{\includegraphics[page=49,width=\paperwidth]{pdfs/lec1}}
\end{frame}
\begin{frame}
\vspace*{-1pt}
\makebox[\linewidth]{\includegraphics[page=50,width=\paperwidth]{pdfs/lec1}}
\end{frame}
\begin{frame}
\vspace*{-1pt}
\makebox[\linewidth]{\includegraphics[page=51,width=\paperwidth]{pdfs/lec1}}
\end{frame}
\begin{frame}
\vspace*{-1pt}
\makebox[\linewidth]{\includegraphics[page=52,width=\paperwidth]{pdfs/lec1}}
\end{frame}
\begin{frame}
\vspace*{-1pt}
\makebox[\linewidth]{\includegraphics[page=53,width=\paperwidth]{pdfs/lec1}}
\end{frame}
\begin{frame}
\vspace*{-1pt}
\makebox[\linewidth]{\includegraphics[page=54,width=\paperwidth]{pdfs/lec1}}
\end{frame}
\begin{frame}
\vspace*{-1pt}
\makebox[\linewidth]{\includegraphics[page=55,width=\paperwidth]{pdfs/lec1}}
\end{frame}
\begin{frame}
\vspace*{-1pt}
\makebox[\linewidth]{\includegraphics[page=56,width=\paperwidth]{pdfs/lec1}}
\end{frame}
\begin{frame}
\vspace*{-1pt}
\makebox[\linewidth]{\includegraphics[page=57,width=\paperwidth]{pdfs/lec1}}
\end{frame}
\begin{frame}
\vspace*{-1pt}
\makebox[\linewidth]{\includegraphics[page=58,width=\paperwidth]{pdfs/lec1}}
\end{frame}
\begin{frame}
\vspace*{-1pt}
\makebox[\linewidth]{\includegraphics[page=59,width=\paperwidth]{pdfs/lec1}}
\end{frame}
\begin{frame}
\vspace*{-1pt}
\makebox[\linewidth]{\includegraphics[page=60,width=\paperwidth]{pdfs/lec1}}
\end{frame}
\begin{frame}
\vspace*{-1pt}
\makebox[\linewidth]{\includegraphics[page=61,width=\paperwidth]{pdfs/lec1}}
\end{frame}
\begin{frame}
\vspace*{-1pt}
\makebox[\linewidth]{\includegraphics[page=62,width=\paperwidth]{pdfs/lec1}}
\end{frame}
\begin{frame}
\vspace*{-1pt}
\makebox[\linewidth]{\includegraphics[page=63,width=\paperwidth]{pdfs/lec1}}
\end{frame}
\begin{frame}{BFS- האלגוריתם}
\LR{
\fbox{
	\parbox{10cm}{
		
		{\sc color[$u$]=white for all $u\in V$\\ } 
		{\sc BFS($s$) }
		\begin{itemize}
			\item[] color[$s$] = gray
			\item[] $d[s] = 0$
			\item[] ENQUEUE($Q,s$)
			\item[] WHILE $Q$ not empty:
			\begin{itemize}
				\item[] DEQUEUE($Q,u$)
				\item[] FOR $(u,v)\in E$ DO
				\begin{itemize}
					\item[] IF color[$v$] = white THEN
					\item[] ~~~~color[$v$] = gray
					\item[] ~~~~$d[v] = d[u] + 1$
					\item[] ~~~~parent[$v$] = u
					\item[] ~~~~ENQUEUE($Q,v$)
					\item[] color[$u$] = black
				\end{itemize}
			\end{itemize}	
		\end{itemize}
}}
}
\vspace*{\baselineskip}
\end{frame}
\begin{frame}{שימושים של BFS}
\pause
\begin{itemize}[<+->]
	\item האם יש מסלול מ-
	\E{$u$}
	ל-
	\E{$v$}?
	\item מה ארכו של המסלול הקצר ביותר
	\item מציאת רכיבי קשירות בגרף לא מכוון
\end{itemize}	
\end{frame}
\begin{frame}{BFS לגרף לא קשיר}
\LR{
\fbox{
	\parbox{8cm}{ 
		FOR each vertex $u \in V $ DO
		\begin{itemize}
			\item[] IF color[$u$] = white THEN BFS($u$)
		\end{itemize}
		OD
}}
}
\end{frame}
\begin{frame}
\centering
\fontsize{16pt}{7.2}\selectfont
100011101 מתחלק ב7?
\end{frame}
\part{תכנון אלגוריתם באמצעות רדוקציה}
\frame{\partpage}
\begin{frame}{דוגמא 1}
\textbf{הגדרה}
יהי  \E{$G=(V,E)$} גרף לא מכוון ויהיו \E{$u,v\in V$} זוג קודקודים ב- \E{$G$}.  \pause נסמן ב-  \E{$d(u,v)$} את אורך המסלול המינימאלי בצלעות מקודקוד \E{$u$} לקודקוד \E{$v$}.  \pause בנוסף, נגדיר כי \E{$d(u,v)=\infty$}   אם לא קיים מסלול מ-u ל- v ב- G.   \pause 

\textbf{הגדרת בעיית \E{SP (Single Source Shortest Paths)}:}\\ \pause
יהי  \E{$G=(V,E)$} גרף לא מכוון ו- \E{$s\in V$} קודקוד ב- \E{$G$}. \pause מצא \E{$d(s,v) $} לכל 
\E{$v\in V$}. 

 \pause
\textbf{הגדרת בעיית \E{SEP (Single Source Shortest Even Paths)}:}\\ \pause
יהי  \E{$G=(V,E)$} גרף לא מכוון ו- \E{$s\in V$} קודקוד ב- \E{$G$}.  \pause מצא את אורך המסלול (לפי צלעות) הזוגי המינאמלי מ-  \E{$s$} לכל 
\E{$v\in V$} או \E{$\infty$} אם לא קיים כזה. 
\end{frame}
\begin{frame}{}
\begin{itemize}[<+->]
	\item איך פותרים את SP?
	\item בהינתן של-SP יש אלגוריתם כיצד נבנה אלגוריתם עבור SEP?
\end{itemize}
\end{frame}
\section{שאלה 1 -  הראו אלגוריתם רדוקציה מ-SEP ל-SP }
%\begin{frame}{הגדרה פורמאלית ל- "רדוקציה":}
%\pause
%תהיינה A ו-B זוג בעיות נתונות. \\\pause
%רדוקציה מ- A  לבעיה B היא  זוג פונקציות f,g, כך ש:
%\\\pause 
%\begin{itemize}[<+->]
%	\item f היא פונקצית המרת הקלט, המעבירה מופע של בעיה A למופע של בעיה B. 
%	\item g היא פונקצית המרת הפלט, המעבירה פתרון של בעיה B לפתרון של בעיה A.
%
%	\item \textbf{הגדרת נכונות}-\pause עבור מופע a לבעיה A, \\ \pause
%	אם \E{B(f(a))}  הוא פתרון עבור המופע \E{f(a)} תחת בעיה B \\ \pause
%	אזי \E{g(B(f(a)))}  הוא פתרון למופע a תחת בעיה A. 
%
%\end{itemize}
%\end{frame}
%\begin{frame}{הוכחת נכונות רדוקציה}
%כדי להוכיח את נכונות הרדוקציה, יש להוכיח שהאלגוריתם הבא פותר את הבעיה A:
% \pause
%\begin{enumerate}[<+->]
%	\item 	עבור מופע a לבעיה A, חשב את \E{f(a)}.
%	\item עבור המופע \E{f(a) } לבעיה B, חשב את הפיתרון b.
%	\item החזר את \E{g(b)} להיות הפתרון של A.
%\end{enumerate}
%
%\end{frame}
%\begin{frame}{הערות}
%\begin{itemize}[<+->]
%	\item 	הפונקציה  f  נקראת פונקצית המרת הקלט.
%	
%	
%	\item 	הפונקציה  g  נקראת פונקצית המרת הפלט.
%
%	\item נתייחס לאלגוריתם שפותר את בעיה B כאל "\textbf{קופסה שחורה}", מבלי להניח דבר על אופן פעולתו מלבד העובדה שהוא אכן פותר נכון את B לכל מופע.
%	\item לעיתים נשתמש בקופסה השחורה כמה פעמים.
%\end{itemize}
%\end{frame}
%\begin{frame}{}
%\begin{figure}
%	\centering
%	\includegraphics[width=\linewidth]{imgs/screenshot002.jpg}	
%\end{figure}
%\end{frame}
%\begin{frame}{שאלה}
%בדוגמה הקודמת, מהי פונקצית הרדוקציה, מהי הקופסה השחורה, ומהי פונקצית תרגום הפלט?
%\end{frame}
\begin{frame}{חישוב זמן ריצה לאלגוריתם רדוקציה}
כפי שראינו ישנם שלושה רכיבים הדורשים זמן חישוב בעת הרצת האלגוריתם: 
\pause
\begin{itemize}[<+->]
	\item חישוב תרגום הקלט, 
	\item הפעלת האלגוריתם לפתרון הבעיה המתורגמת, 
	\item וחישוב תרגום הפלט. 
\end{itemize}
\end{frame}
\begin{frame}{חישוב זמן ריצה לאלגוריתם רדוקציה}
\pause
\begin{itemize}[<+->]
	\item זמן חישוב האלגוריתם המבוסס על הרדוקציה הינו הזמן המקסימאלי הדרוש מבין השלושה.
	\item נתייחס לאלגוריתם שפותר את בעיה המתורגמת כאל "\textbf{קופסה שחורה}", מבלי להניח דבר על אופן פעולתו מלבד העובדה שהוא אכן פותר נכון את הבעיה השניה לכל מופע.
	\item נשים לב שזמן הריצה של הקופסא השחורה הוא על גודל הקלט החדש (הפלט של ממיר הקלט).
\end{itemize}
\end{frame}
\section{שאלה 2 - אלגוריתם רדוקציה מ-SP ל-SEP }
\begin{frame}{תרגיל למחשבה}
\textbf{הגדרת בעיית \E{SOP (Single Source Shortest Odd Paths)}:}\\
יהי  \E{$G=(V,E)$} גרף לא מכוון ו- \E{$s\in V$} קודקוד ב- \E{$G$}. מצא את אורך המסלול (לפי צלעות) \textbf{האי-זוגי }המינאמלי מ-  \E{$s$} לכל 
\E{$v\in V$} או \E{$\infty$} אם לא קיים כזה. 
\end{frame}



\end{document}
