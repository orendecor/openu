%! TEX program = lualatex
% find: (\$((?!\$).)+\$)
%replace: \E{\1}

%%%%%%%%%%
% BEAMER %
%%%%%%%%%%

\RequirePackage{luatex85}

\PassOptionsToPackage{luatex}{hyperref}

%\documentclass[luatex]{beamer}
\documentclass[handout]{beamer}

\usetheme{metropolis}

	\newcommand{\bigq}{\mathbb{Q}}
\newcommand{\bb}{\bfseries\large}
\newcommand{\class}{\mathcal}
\newcommand{\mbi}[1]{\textbf{\em{#1}}}
\newcommand{\bi}[1]{\textbf{\textit{#1}}}
\newcommand{\ita}[1]{\textit{#1}}
\newcommand{\p}{\cdot}
\newcommand{\req}{\emptyset}
\newcommand{\init}{\mathcal{J}}
\newcommand{\power}[1]{\class{P}(#1)}
\newcommand{\bigpower}[2]{\class{P}^{#1}(#2)}
\newcommand{\symetric}{\bigtriangleup}
\newcommand{\gx}[3]{\Gamma_{(#1,#2)}(#3)}
\newcommand{\bd}[3]{\Delta_{(#1,#2)}(#3)}
\newcommand{\bp}[3]{\Phi_{(#1,#2)}(#3)}
\newcommand{\eab}{\eta_{\alpha \beta}}
\newcommand{\sab}{\sigma_{\alpha\beta}}
\newcommand{\dab}{\class{D}_{\alpha\beta}}
\newcommand{\cab}{\class{C}_{\alpha\beta}}
\newcommand{\C}[1]{\class{#1}}
\newcommand{\ol}{\overline}
\newcommand{\into}{\longrightarrow}
\newcommand{\ifff}{\Longleftrightarrow}
\newcommand{\imply}{\Rightarrow}
\newcommand{\since}{\Longleftarrow}
\newcommand{\la}{\langle}
\newcommand{\ra}{\rangle}
\newcommand{\andd}{\wedge}
\newcommand{\orr}{\vee}
\newcommand{\ft}{(\C F \C T)_Q}
\newcommand{\pathopen}[3]{#1_{(\ol{#2},#3)}}
\newcommand{\fpi}[3]{#1_*(\Pi(#2,#3))}
\newcommand{\xpath}[2]{(\ol {#1},#2)}
\newcommand{\qn}[2]{\Lambda(#1,#2)}
\newcommand{\jh}{\class J \class H}
\newcommand{\bos}{\boldsymbol}
\newcommand{\f}{\mathbb}
\newcommand{\rss}{\la\psi_i:i\in \lambda\ra}
\newcommand{\ito}[1]{\overset{#1}{\longrightarrow}}
\newcommand{\pcomplex}[2]{\class #1(\class #2)}
\newcommand{\then}{\Rightarrow}
\newcommand{\ud}{{ {up-down-automaton}} }

\newcommand{\map}[1]{\underset{#1}{\mapsto}}
\newcommand{\mapi}[1]{\underset{#1}{\overset{i}{\mapsto}}}
\newcommand{\mapstar}[1]{\underset{#1}{\overset{*}{\mapsto}}}
\newcommand{\mapk}[2]{\underset{#1}{\overset{#2}{\mapsto}}}

\newcommand{\eql}{\equiv_L}
\newcommand{\der}[1]{\underset{#1}{\imply}}
\newcommand{\deri}[1]{\underset{#1}{\overset{i}{\imply}}}
\newcommand{\derstar}[1]{\underset{#1}{\overset{*}{\imply}}}
\newcommand{\derk}[2]{\underset{#1}{\overset{#2}{\imply}}}
\newcommand{\blank}{\bar{b}}
\newcommand{\R}[1]{#1}
\newcommand{\say}[1]{\color{blue}{#1}\color{black}}



% Fix the title separator
\makeatletter
\setbeamertemplate{title separator}{
	\bgroup
	\bodydir TLT
	\begin{tikzpicture}
	\fill[fg] (0,0) rectangle (\textwidth, \metropolis@titleseparator@linewidth);
	\end{tikzpicture}%
	\egroup
	\par%
}
\makeatother

% Fix the progress bar and make it progress from right to left
\makeatletter
\setbeamertemplate{progress bar in section page}{
	\setlength{\metropolis@progressonsectionpage}{%
		\textwidth * \ratio{\insertframenumber pt}{\inserttotalframenumber pt}%
	}%
	\bgroup
	\bodydir TLT
	\begin{tikzpicture}
	\fill[bg] (0,0) rectangle (\textwidth, \metropolis@progressonsectionpage@linewidth);
	\fill[fg] (\textwidth,0) rectangle (\textwidth-\metropolis@progressonsectionpage, \metropolis@progressonsectionpage@linewidth);
	\end{tikzpicture}%
	\egroup
}
\makeatother

% A more logical title page, juxtaposing `institute` to `author`. This modification is not necessary for Hebrew support.
\makeatletter
\setbeamertemplate{title page}{
	\begin{minipage}[b][\paperheight]{\textwidth}
		\ifx\inserttitlegraphic\@empty\else\usebeamertemplate*{title graphic}\fi
		\vfill%
		\ifx\inserttitle\@empty\else\usebeamertemplate*{title}\fi
		\ifx\insertsubtitle\@empty\else\usebeamertemplate*{subtitle}\fi
		\usebeamertemplate*{title separator}
		\ifx\beamer@shortauthor\@empty\else\usebeamertemplate*{author}\fi
		\ifx\insertinstitute\@empty\else\usebeamertemplate*{institute}\fi
		\ifx\insertdate\@empty\else\usebeamertemplate*{date}\fi
		\vfill
		\vspace*{1mm}
	\end{minipage}
}
\setbeamertemplate{author}{\vspace*{2em}\insertauthor\par\vspace*{0.25em}}
\setbeamertemplate{institute}{\insertinstitute\par}
\setbeamertemplate{date}{\vspace*{3mm}\insertdate\par}
\makeatother




%%%%%%%%%
% Babel %
%%%%%%%%%

\usepackage[nil,bidi=basic-r]{babel}
\babelprovide[import=he,main]{hebrew}
\babelprovide[import=en-GB]{english}

% For some reason Babel’s `\babelfont` doesn’t work
\setsansfont[Script=Hebrew]{Open Sans Hebrew}
\setmonofont{Fira Mono}
\renewcommand{\H}[1]{\foreignlanguage{hebrew}{\fontspec[Script=Hebrew]{Open Sans Hebrew}#1}}
\newcommand{\E}[1]{\foreignlanguage{english}{\fontspec{Open Sans}#1}}
\newcommand{\LR}[1]{{‏\textdir TLT #1}}




%%%%%%%%
% MISC %
%%%%%%%%

\usepackage{metalogo, fancyvrb}
\usepackage{cancel}
\newcommand{\smallurl}[1]{{\footnotesize\url{#1}}}



%%%%%%%%%%%%
% DOCUMENT %
%%%%%%%%%%%%

\begin{document}
	\title{FFT - התמרת פורייה המהירה}
	\subtitle{מפגש 8}
	%\author{}
	%\institute{}
	\date{}
	
	
	\frame{\titlepage}
	\begin{frame}{	בוא נזכר מה התחלנו שבוע שעבר}
	\pause
	\begin{itemize}[<+->]
		\item נזכרנו שקונבולוציה אנחנו מכירים כבר ממזמן - זה בעצם הפעולה שאנחנו עושים שאנחנו מכפילים שני פולינומים
		\item למעשה - לחשב את הקונבולוציה של שני וקטורים שקול ללחשב הכפלה של שני פולינומים
	\end{itemize}
\end{frame}

\begin{frame}{קונבולוציה - הגדרה}
\pause
\begin{itemize}[<+->]
	\item בהינתן שני וקטורים
	$$A=(a_0,a_1,\ldots,a_n)$$
		$$B=(b_0,b_1,\ldots,b_m)$$
	\item קונבולוציה של \E{$A$} עם \E{$B$} מסומנת ב: \E{$A*B = C$} 
	\item עבור וקטור 
	\E{$C$} באורך \E{$1+n+m$} המוגדר כך:
	$$c_i=\sum_{j,k:j+k=i}a_j\cdot b_k$$
\end{itemize}
\end{frame}

\begin{frame}{ייצוג פולינומים}
בהינתן פולינום מדרגה \E{$n$} ניתן לייצג אותו:
\pause
\begin{itemize}[<+->]
	\item ע"י \E{$n+1$}  מקדמים 
	\item ע"י \E{$n+1$} נקודות
\end{itemize}
\pause
\textbf{המשפט היסודי של האלגברה:} דרך \E{$n+1$} נקודות עובר פולינום יחיד ממעלה \E{$n$}
\end{frame}
\begin{frame}{ההבדלים בין הייצוגים}
\pause
\begin{itemize}[<+->]
	\item קבלת הערך בנקודה
	\item מכפלת שני פולינומים
	\pause
	\begin{itemize}[<+->]
		\item בייצוג מקדמים - דורש זמן ריבועי
		\item בייצוג בנקודות - אם יש לנו \E{$2n+1$} נקודות של כל פולינום ניתן לחשב בזמן לינארי 
	\end{itemize}
\end{itemize}
\end{frame}
\begin{frame}{סקיצה לאלגוריתם לחישוב קונבולוציה}
 על מנת לחשב את \E{$C=A*B$}
\pause
\begin{enumerate}[<+->]
	\item נחשב את \E{$A$} ו-\E{$B$} ב-\E{$2n$} נקודות
	\item נכפול נקודתית את ערכי \E{$A$} ו-\E{$B$}
	\item נחשב את מקדמי הפולינום \E{$C$} מתוך הנקודות
\end{enumerate}
\pause
כיצד נעבור מייצוג לייצוג בצורה יעילה? (מהירה יותר מ-\E{$\Theta(n^2)$}).
\end{frame}
\section{דוגמא לחישוב פולינום ב-n נקודות}
\begin{frame}{אז מצאנו שיטה הפרד ומשול לחישוב פולינום בנקודות}
\pause
\begin{itemize}[<+->]
	\item בהינתן פולינום המיוצג על ידי מקדמים: \E{$p=(a_0,a_1,\ldots,a_{n-1})$}
	\item בנה שני פולינומים:
	$$
	p_{odd}=(a_1,a_3,\ldots,a_{n-1})$$
	$$	p_{even} =(a_0,a_2,\ldots,a_{n-2})
	$$
	\item חשב רקורסיבית \E{$\frac{n}{2}$} ערכים של \E{$p_{odd}$} ו- \E{$p_{even}$} בנקודות:
	$$t_1^2,t_2^2,t_{\frac{n}{2}}^2$$
	\item לכל \E{$z\in \{t_i,-t_i:1\le i \le \frac{n}{2}\}$} מצא את ערכי \E{$p(z)$} ע"י הנוסחא:
	$$p(x)=p_{even}(x^2)+xp_{odd}(x^2)$$
	סה"כ \E{$n$} נקודות.
\end{itemize}
\end{frame}
\begin{frame}{תרגיל}
חשבו את הפולינום המתאים לקורדינטות הבאות \E{$p(x)=x^3+5x+8$} ב-4 הנקודות הבאות \E{$(1,-1,2,-2)$}
\pause
\begin{itemize}[<+->]
	\item תחילה חשבו את \E{$p_{even}$} ו- \E{$p_{odd}$}.
	\item לאחר מכן חשבו את הפולינום המוקטנים בנקודות המרובעות ע"י הצבה
	\item מצאו את 4 הנקודות הנדרשות ע"י הנוסחא:
	$$p(x)=p_{even}(x^2)+xp_{odd}(x^2)$$
\end{itemize}
\end{frame}
\begin{frame}{חשיבות לבחירת הנקודות}
\pause
\begin{itemize}[<+->]
	\item אם היינו רוצים רק לחשב פולינום ב-\E{$n$} נקודות כל נקודה $t_i\ne 0$ הייתה עובדת לנו
	\item אנחנו רוצים גם לאחר מכן לשחזר את מקדמי פולינום התוצאה (C) מתוך הנקודות
	\item לשם כך אנחנו צריכים לבחור נקודות חכמות שקל יהיה לשחזר מתוכן את הפולינום
\end{itemize}
\end{frame}

\begin{frame}{חזרה מהירה על מורכבים}
\pause
\begin{itemize}[<+->]
	\item \E{$i^2=-1$}
	\item לפולינום מדרגה \E{$n$} יש \E{$n$} שורשים מעל המורכבים
	\item לכן למשוואה הזו $$x^n-1$$ ישנם בדיוק $n$ שורשים, הנקראים שורשי היחידה
	\item אלו בדיוק n הנקודות שנבחר 
\end{itemize}
\end{frame}
\begin{frame}{שורשי היחידה}
\E{$x^n-1$}
\pause
\begin{itemize}[<+->]
	\item כאשר \E{$n=2$} מה אלו שורשי היחידה?
	\item כאשר \E{$n=4$} שורשי היחידה הם:
	$$1,i,-1,-i$$
	\item שורש יחידה כללי מסדר \E{$n$} מסומן ב-\E{$\omega_{j,n}$} ומוגדר כך:
	$$\omega_{j,n} = e^\frac{2\pi i\cdot j}{n},\quad j=0,1,\ldots, n-1$$
\end{itemize}
\end{frame}

\begin{frame}{שורשי היחידה}

\pause
\begin{itemize}[<+->]
	\item בואו נוודא שאכן \E{$\omega_{j,n}$} זהו שורש של \E{$x^n-1$}:
	$$\omega_{j,n}^n = (e^\frac{2\pi i\cdot j}{n})^n =e^\frac{2\pi i\cdot j\cdot n}{n}=e^{2\pi i\cdot j} = (e^{2\pi i})^j = 1^j = 1$$


\end{itemize}
\pause	
* במעבר הלפני האחרון השתמשתנו בזהות אוילר: \E{$e^{2\pi i} = 1$}
\end{frame}
 \begin{frame}{שורש יחידה פרימיטיבי}
\pause
\begin{itemize}[<+->]
	\item שורש יחידה פרימיטיבי הוא כזה שלא הופך לאחד באף חזקה פרט ל-\E{$n$}.
	\item פורמלית: \E{$\omega_{j,n}$} הוא שורש יחידה פרימיטיבי אם
	$$\omega_{j,n}^k\ne 1, \quad \forall 0< k<n$$
	\item כאשר ידוע מיהו \E{$n$} מסמנים אותו פשוט כך: \E{$\omega$}
	\item נבחין שלכל \E{$0< k<n$}: 
	$$\omega^k= (e^\frac{2\pi i\cdot j}{n})^k=e^\frac{2\pi i\cdot j\cdot k}{n}\ne 1$$
\textbf{	ולכן \E{$\omega^k$} הוא גם שורש יחידה מסדר \E{$n$} (לא פרימיטיבי).}
\end{itemize}
\end{frame}
\begin{frame}{אותו תרגיל בשינוי קטן}
חשבו את הפולינום המתאים לקורדינטות הבאות \E{$p(x)=x^3+5x+8$} ב-4 הנקודות הבאות \E{$(1,i,-1,-i)$}
\pause
\begin{itemize}[<+->]
	\item תחילה חשבו את \E{$p_{even}$} ו- \E{$p_{odd}$}.
	\item לאחר מכן חשבו את הפולינום המוקטנים בנקודות המרובעות ע"י הצבה
	\item מצאו את 4 הנקודות הנדרשות ע"י הנוסחא:
	$$p(x)=p_{even}(x^2)+xp_{odd}(x^2)$$
\end{itemize}
\pause
מה שלמעשה עשינו כאן נקרא התמרת פורייה הבדידה (DFT) והיא למעשה הפעולה שאלגוריתם \E{$FFT(p(x),\omega)$} מבצע. כאן מה שלמעשה עשינו הוא הפעלה של \E{$FFT((8,5,0,1),i)$}.
\end{frame}
\begin{frame}{השלב האחרון שנותר לנו לממש באלגוריתם}
על מנת לחשב את \E{$C=A*B$}
\pause
\begin{enumerate}[<+->]
	\item נחשב את \E{$A$} ו-\E{$B$} ב-\E{$2n$} נקודות (שורשי היחידה מסדר \E{$2n$})
	\item נכפול נקודתית את ערכי \E{$A$} ו-\E{$B$}
	\item\textbf{ נחשב את מקדמי הפולינום \E{$C$} מתוך הנקודות}
\end{enumerate}
\end{frame}
\begin{frame}{קסם שלא נסביר}
\pause
\begin{itemize}[<+->]
	\item על מנת לחשב את \E{$p(x)$} ב-\E{$n$} נקודות הפעלנו את אלגוריתם 
	$$FFT((a_0,a_1,\ldots,a_{n-1}),\omega)$$ 
	כאשר \E{$(a_0,a_1,\ldots,a_{n-1})$} הן מקדמי \E{$p(x)$} ו-\E{$\omega$} הוא שורש יחידה כלשהו מסדר $n$.
	\item כעת אם אנחנו יודעים את פולינום כלשהו ב-$n$ נקודות שהן שורשי היחידה, הפעלה של:
		$$FFT(\cdot,\omega^{-1})$$
		משחזרת את מקדמי הפולינום. (לאחר חלוקה ב-\E{$n$})
		\item מי זה \E{$\omega^{-1}$}?
\end{itemize}
\end{frame}
\begin{frame}{הערות}
\pause
\begin{itemize}[<+->]
	\item שימו לב שזו הסיבה העיקרית לבחירה שלנו להעריך את
	\E{$p(x)$} 
	בשורשי היחידה. 
	\item 	אם היינו מעריכים את  \E{$p(x)$}  בסתם נקודות הכל היה עובד אך לא היינו מצליחים לשחזר חזרה את המקדמים מתוך הנקודות בשלב האחרון באלגוריתם.
	\item אתם מוזמנים לקרוא בספר את ההוכחה "לקסם" - היא אלגברית לחלוטין
	
	(מראים שם שהתמרת פורייה היא העתקה לינארית ומוכיחים שהתמרת פורייה עם \E{$\omega^{-1}$} היא בדיוק ההעתקה ההפוכה לה.)
\end{itemize}
\end{frame}

\begin{frame}{תרגיל}
חשבו את \E{$FFT(\cdot,\omega^{-1})$} על התוצאה שקיבלתם בתרגיל הקודם.\\
(תזכורת לתרגיל: חשבו את הפולינום המתאים לקורדינטות הבאות \E{$p(x)=x^3+5x+8$} ב-4 הנקודות הבאות \E{$(1,i,-1,-i)$})

\end{frame}
\begin{frame}{תרגיל}
\pause
\begin{itemize}[<+->]
	\item נגדיר \E{$P_A=x^2\quad P_B=3x+8$}
	\item חשבו את \E{$C=P_A\cdot P_B$} ע"י הרצת:
	\pause
	\begin{itemize}[<+->]
		\item חשבו  \E{$FFT_4(A),FFT_4(B)$} ומצאו את הפולינומים ב-4 שורשי היחידה מסדר 4
		\item הכפילו את הערכה של A,B בשורשי היחידה על מנת לקבל הערכה של פולינום \E{$C$} ב4 שורשי היחידה
		\item מצאו את מקדמי \E{$C$} ע"י הפעלה של \E{$FFT(\cdot,\omega^{-1})$} על התוצאה מהסעיף הקודם
	\end{itemize}
	
\end{itemize}
\end{frame}
\begin{frame}{שאלה מתוך מבחן}
\pause
\begin{itemize}[<+->]
	\item נתונות שתי קבוצות של \E{$n$} מספרים טבעיים קטנים מ-\E{$16n$}
	\item \E{$A,B\subseteq \{1,2,...16n\}$}
	\item  חשבו את אוסף כל הזוגות
	$$\{x+y:x\in A,y\in B\}$$
	\item דוגמא 
	$$A=\{1,10\},\quad B=\{1,3\} \quad \{2,4,11,13\}$$
\end{itemize}
\end{frame}

\end{document}
