%! TEX program = lualatex
% find: (\$((?!\$).)+\$)
%replace: \E{\1}

%%%%%%%%%%
% BEAMER %
%%%%%%%%%%

\RequirePackage{luatex85}

\PassOptionsToPackage{luatex}{hyperref}

\documentclass[luatex]{beamer}
%\documentclass[handout]{beamer}

\usetheme{metropolis}

	\newcommand{\bigq}{\mathbb{Q}}
\newcommand{\bb}{\bfseries\large}
\newcommand{\class}{\mathcal}
\newcommand{\mbi}[1]{\textbf{\em{#1}}}
\newcommand{\bi}[1]{\textbf{\textit{#1}}}
\newcommand{\ita}[1]{\textit{#1}}
\newcommand{\p}{\cdot}
\newcommand{\req}{\emptyset}
\newcommand{\init}{\mathcal{J}}
\newcommand{\power}[1]{\class{P}(#1)}
\newcommand{\bigpower}[2]{\class{P}^{#1}(#2)}
\newcommand{\symetric}{\bigtriangleup}
\newcommand{\gx}[3]{\Gamma_{(#1,#2)}(#3)}
\newcommand{\bd}[3]{\Delta_{(#1,#2)}(#3)}
\newcommand{\bp}[3]{\Phi_{(#1,#2)}(#3)}
\newcommand{\eab}{\eta_{\alpha \beta}}
\newcommand{\sab}{\sigma_{\alpha\beta}}
\newcommand{\dab}{\class{D}_{\alpha\beta}}
\newcommand{\cab}{\class{C}_{\alpha\beta}}
\newcommand{\C}[1]{\class{#1}}
\newcommand{\ol}{\overline}
\newcommand{\into}{\longrightarrow}
\newcommand{\ifff}{\Longleftrightarrow}
\newcommand{\imply}{\Rightarrow}
\newcommand{\since}{\Longleftarrow}
\newcommand{\la}{\langle}
\newcommand{\ra}{\rangle}
\newcommand{\andd}{\wedge}
\newcommand{\orr}{\vee}
\newcommand{\ft}{(\C F \C T)_Q}
\newcommand{\pathopen}[3]{#1_{(\ol{#2},#3)}}
\newcommand{\fpi}[3]{#1_*(\Pi(#2,#3))}
\newcommand{\xpath}[2]{(\ol {#1},#2)}
\newcommand{\qn}[2]{\Lambda(#1,#2)}
\newcommand{\jh}{\class J \class H}
\newcommand{\bos}{\boldsymbol}
\newcommand{\f}{\mathbb}
\newcommand{\rss}{\la\psi_i:i\in \lambda\ra}
\newcommand{\ito}[1]{\overset{#1}{\longrightarrow}}
\newcommand{\pcomplex}[2]{\class #1(\class #2)}
\newcommand{\then}{\Rightarrow}
\newcommand{\ud}{{ {up-down-automaton}} }

\newcommand{\map}[1]{\underset{#1}{\mapsto}}
\newcommand{\mapi}[1]{\underset{#1}{\overset{i}{\mapsto}}}
\newcommand{\mapstar}[1]{\underset{#1}{\overset{*}{\mapsto}}}
\newcommand{\mapk}[2]{\underset{#1}{\overset{#2}{\mapsto}}}

\newcommand{\eql}{\equiv_L}
\newcommand{\der}[1]{\underset{#1}{\imply}}
\newcommand{\deri}[1]{\underset{#1}{\overset{i}{\imply}}}
\newcommand{\derstar}[1]{\underset{#1}{\overset{*}{\imply}}}
\newcommand{\derk}[2]{\underset{#1}{\overset{#2}{\imply}}}
\newcommand{\blank}{\bar{b}}
\newcommand{\R}[1]{#1}
\newcommand{\say}[1]{\color{blue}{#1}\color{black}}



% Fix the title separator
\makeatletter
\setbeamertemplate{title separator}{
	\bgroup
	\bodydir TLT
	\begin{tikzpicture}
	\fill[fg] (0,0) rectangle (\textwidth, \metropolis@titleseparator@linewidth);
	\end{tikzpicture}%
	\egroup
	\par%
}
\makeatother

% Fix the progress bar and make it progress from right to left
\makeatletter
\setbeamertemplate{progress bar in section page}{
	\setlength{\metropolis@progressonsectionpage}{%
		\textwidth * \ratio{\insertframenumber pt}{\inserttotalframenumber pt}%
	}%
	\bgroup
	\bodydir TLT
	\begin{tikzpicture}
	\fill[bg] (0,0) rectangle (\textwidth, \metropolis@progressonsectionpage@linewidth);
	\fill[fg] (\textwidth,0) rectangle (\textwidth-\metropolis@progressonsectionpage, \metropolis@progressonsectionpage@linewidth);
	\end{tikzpicture}%
	\egroup
}
\makeatother

% A more logical title page, juxtaposing `institute` to `author`. This modification is not necessary for Hebrew support.
\makeatletter
\setbeamertemplate{title page}{
	\begin{minipage}[b][\paperheight]{\textwidth}
		\ifx\inserttitlegraphic\@empty\else\usebeamertemplate*{title graphic}\fi
		\vfill%
		\ifx\inserttitle\@empty\else\usebeamertemplate*{title}\fi
		\ifx\insertsubtitle\@empty\else\usebeamertemplate*{subtitle}\fi
		\usebeamertemplate*{title separator}
		\ifx\beamer@shortauthor\@empty\else\usebeamertemplate*{author}\fi
		\ifx\insertinstitute\@empty\else\usebeamertemplate*{institute}\fi
		\ifx\insertdate\@empty\else\usebeamertemplate*{date}\fi
		\vfill
		\vspace*{1mm}
	\end{minipage}
}
\setbeamertemplate{author}{\vspace*{2em}\insertauthor\par\vspace*{0.25em}}
\setbeamertemplate{institute}{\insertinstitute\par}
\setbeamertemplate{date}{\vspace*{3mm}\insertdate\par}
\makeatother




%%%%%%%%%
% Babel %
%%%%%%%%%

\usepackage[nil,bidi=basic-r]{babel}
\babelprovide[import=he,main]{hebrew}
\babelprovide[import=en-GB]{english}

% For some reason Babel’s `\babelfont` doesn’t work
\setsansfont[Script=Hebrew]{Open Sans Hebrew}
\setmonofont{Fira Mono}
\renewcommand{\H}[1]{\foreignlanguage{hebrew}{\fontspec[Script=Hebrew]{Open Sans Hebrew}#1}}
\newcommand{\E}[1]{\foreignlanguage{english}{\fontspec{Open Sans}#1}}
\newcommand{\LR}[1]{{‏\textdir TLT #1}}




%%%%%%%%
% MISC %
%%%%%%%%

\usepackage{metalogo, fancyvrb}
\usepackage{cancel}
\newcommand{\smallurl}[1]{{\footnotesize\url{#1}}}



%%%%%%%%%%%%
% DOCUMENT %
%%%%%%%%%%%%

\begin{document}
	\title{קודי הופמן ודחיסת נתונים}
	\subtitle{מפגש 6}
	%\author{}
	%\institute{}
	\date{}
	
	
	\frame{\titlepage}
	\begin{frame}{שבוע שעבר ראינו}
	\pause
	
	\begin{itemize}[<+->]
		\item בעיית עץ הפורש המינימלי
		\begin{itemize}[<+->]
			\item אלגוריתם של פרים
			\item האלגוריתם של קרוסקל
		\end{itemize}
	\end{itemize}
\end{frame}

\begin{frame}{התוכנית להיום}
\pause
\begin{itemize}[<+->]
	\item דחיסת נתונים באמצעות קידוד הופמן
	\begin{itemize}[<+->]
		\item הצגת הבעיה
		\item הגדרות
		\item האלגוריתם של הופמן
	\end{itemize}
\end{itemize}
\end{frame}
\begin{frame}
\begin{figure}
	\centering
	\includegraphics[width=\linewidth]{imgs/telegraph}
\end{figure}


\end{frame}
\begin{frame}{מטרות התקשורת}
\pause
\begin{itemize}[<+->]
	\item תקשורת יעילה
	\item ניתנת לפענוח
\end{itemize}
\end{frame}
\begin{frame}{תקשורת יעילה}

\centering \E{i love algorithm course. }
\pause
\begin{itemize}[<+->]
	\item $f(space(= 4/24$
	\item $f(o(=3/24$
	\item $f(i( = 2/24$
	\item ...
\end{itemize}
\end{frame}
\begin{frame}{קידוד אפשרי}

\pause
\begin{itemize}[<+->]
	\item הקידוד של space יהיה - 0 
	\item הקידוד של o יהיה - 1
	\item הקידוד של i יהיה  - 01
	\item וכ"ו...
\end{itemize}
\pause
נקבל: 01001101001010110010100100
\pause 
\\איך אפשר לקרוא את זה?
\end{frame}
\begin{frame}{הגדרות וסימונים}
\pause
\begin{itemize}[<+->]
	\item \E{$S$} - א"ב שאנחנו רוצים לקודד
	\item \E{$f$} - פונק' שכיחויות של הא"ב
	$$f : S \rightarrow (0,\infty) $$
	\item \E{$\gamma$} - קידוד תחיליות של א"ב
	$$\gamma : S \rightarrow \{0,1\}^*$$
	\item \E{$ABL(\gamma)$} - מספר ביטים ממוצע לאות בקידוד \E{$\gamma$}.
	$$ABL(\gamma) = \sum_{x\in S} f(x)\cdot |\gamma(x)|$$
\end{itemize}
\end{frame}
\begin{frame}{בעיית קוד תחיליות אופטימלי}
\pause
\begin{itemize}[<+->]
	\item \textbf{הקלט:} קבוצה 
	\E{$S$} ושכיחויות 
	\E{$f:S\rightarrow (0,\infty)$}
	\item \textbf{הפלט:}
	 מיפוי 
	\E{$\gamma : S \rightarrow \{0,1\}^*$}
	\\ כך שלכל
	 \E{$  x, y \in S $}
	   שונים,
	מתקיים תנאי התחיליות:
	\pause
	\begin{itemize}[<+->]
		\item 	 אף אחת מהמחרוזות הבינריות 
		\E{$\gamma(x),\gamma(y)$}
		אינה תחילית של
		האחרת.
	\end{itemize}

	\item \textbf{מטרה: }
	למזער את הסכום
\E{$ABL(\gamma)=\sum_{x\in S} f(x)|\gamma(x)| $}
\end{itemize}
\end{frame}

\section{תכנון האלגוריתם של הופמן}
\begin{frame}{ייצוג קודי תחיליות בעצים בינאריים}
\begin{center}
	\includegraphics[width=1\linewidth]{imgs/htree}
\end{center}
\end{frame}
\begin{frame}{ייצוג קודי תחיליות בעצים בינאריים}
\pause
\begin{itemize}[<+->]
	\item \textbf{משפט: } קידוד הבנוי על בסיס עץ בינארי הוא קוד של תחיליות
	\item \textbf{משפט: } עץ בינארי שמתאים לקוד אופטימלי של תחיליות הוא עץ מלא.
\end{itemize}
\end{frame}
\begin{frame}{הבעיה השקולה}
\textbf{הגדרה: }\E{$ABL(T)$} - ממוצע משוקלל של עומקי העלים בעץ \E{$T$}.
$$ABL(T) = \sum_{x\in S} f(x)\cdot depth_T(x)$$
\pause
\begin{itemize}[<+->]
	\item \textbf{הקלט:} קבוצה 
\E{$S$} ושכיחויות 
\E{$f:S\rightarrow (0,\infty)$}
\item \textbf{הפלט:}
עץ בינארי \E{$T$} המכיל קבוצת עלים \E{$S$}


\item \textbf{מטרה: }
למזער את הסכום
\E{$ABL(T)$}
\end{itemize}
\end{frame}
\begin{frame}{משפט מפתח}
נביט בעץ בינארי 
\E{$T^*$}
 שמתאים לקוד אופטימלי של תחיליות.
יהי
\E{$v$}
קודקוד עמוק ביותר בעץ. אז 
\E{$v$}
 עלה ויש לו שכן
\E{   $w$}
    שהינו גם עלה.
\end{frame}
\begin{frame}{קידוד מלמטה למעלה}
\pause
\begin{itemize}[<+->]
	\item יש לנו שתי עלים אחים עמוקים ביותר - איזה אותיות הן ייצגו?
\end{itemize}
\begin{center}
	\includegraphics[width=4cm]{imgs/htree2}
	\end{center}
\pause
\begin{itemize}[<+->]
	\item \textbf{טענה:} נניח ש-
	\E{$u$}
	ו-
		\E{$v$}
		הם עלים ב-
			\E{$T^*$}
			עץ המותאם לקוד תחיליות אופטימלי.
			אם הם מקיימים
				\E{$depth(u)<depth(v)$},
				אזי התדירות של האות שמוצמדת ל-
					\E{$u$}
					גדולה מהתדירות של האות שמוצמדת ל- 	\E{$v$}
					.
	\item \textbf{מסקנה:} קיים קוד אופטימלי של תחיליות ועץ תואם 
	\E{$T^*$}
	שבו שתי האותיות בעלות השכיחיות הנמוכות ביותר משוייכות לעלים שהם אחים ב-\E{$T^*$}.
	
\end{itemize}
\end{frame}
\begin{frame}{}
\begin{center}
	\includegraphics[width=0.5\linewidth]{huffmann-running-example}
\end{center}

\end{frame}
\begin{frame}{תיאור האלגוריתם}
\pause
ליצירת קוד תחיליות לשפה עם א"ב \E{$S$}: 
\begin{enumerate}[<+->]
	\item אם ב-\E{$S$} יש שתי אותיות או פחות:
	\pause
	\begin{itemize}[<+->]
		\item קבע את הראשונה בקידוד 0 ואת השניה בקידוד 1.
	\end{itemize}
	\item אחרת
	\pause
	\begin{enumerate}[<+->]
		\item יהיו, \E{$x,y$} שני האיברים עם התדירות הנמוכה ביותר ב-\E{$S$}.
		\item צור א"ב חדש \E{$S'$} ע"י מחיקת 
		\E{ $x,y$} 
		והחלפתם באות חדשה \E{$w$} בעלת התדירות \E{$f(x)+f(y)$}.
		\item צור בצורה רקורסיבית קוד תחיליות \E{$\gamma'$} ל-\E{$S'$}, עם עץ \E{$T'$}
		\item  הגדר קוד תחיליות עבור \E{$S$} :בצורה הבאה
		\pause
		\begin{itemize}[<+->]
			\item התחל עם \E{$T'$}
			\item קח את העלה שמתיוג ב-\E{$w$} והוסף לו שתי ילדים המתיוגים כ-\E{$x,y$}.
		\end{itemize}
	\end{enumerate}
\end{enumerate}

\end{frame}
\section{הוכחת נכונות}
\begin{frame}{תרגיל 4.20 מהמדריך}
בהינתן קבוצת תווים
 \E{$S$}
 שהשכיחויות שלהם הן
 \E{$f:S\rightarrow (0,\infty)$}
  , פתחו אלגוריתם ליצירת 􏰌
קוד תחיליות
\E{$\gamma : S \rightarrow \{0,1,2\}$}
 שימזער את
 \E{$\sum_{x\in S} f(x)|\gamma(x)|$}
  . הוכיחו את הנכונות ונתחו את הסיבוכיות. כדי לקצר את התשובה במקצת, אתם יכולים להניח 
  ש־
  \E{$|S|\ge 3$}
   ואי־זוגי.
\end{frame}
\end{document}
